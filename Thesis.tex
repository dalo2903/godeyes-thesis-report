\documentclass[a4paper,12pt,oneside]{ThesisStyle}

%\usepackage[style=verbose]{biblatex}

\usepackage{listings}
\usepackage{color}
\usepackage[utf8]{vietnam}
%\usepackage[monochrome]{color} %enable to make black-while print version
\usepackage{arial}	% same Time New Roma
\usepackage{titlesec}
\usepackage{subcaption}
\usepackage{amsmath}
\titleformat
{\chapter} % command
[display] % shape
{\bfseries\LARGE} % format
{\Large{Chapter} \ \thechapter} % label
{0.5ex} % sep
{
    \rule{\textwidth}{1pt}
    \vspace{1ex}
    \centering
} % before-code
[
\vspace{-0.5ex}%
\rule{\textwidth}{0.3pt}
] % after-code


\titlespacing*{\chapter}{0pt}{-1cm}{1cm}


%\end trong add

\usepackage{tabularx}
\usepackage{multirow}
\usepackage{pdf14}
\usepackage{pgfplots}
\usetikzlibrary{calc}
\usepackage{booktabs}
\usepackage{array}
\usepackage{tikz}
\usepackage{amsmath,amssymb}             % AMS Math
\usepackage[left=3cm,right=2cm,top=2cm,bottom=2cm,includefoot,includehead,headheight=13.6pt]{geometry}

\usepackage{floatrow}
\floatsetup[table]{capposition=top}
\usepackage{multirow}
\usepackage[english]{babel}
\usepackage[format=plain,indention=1em]{caption}
\usepackage{float}
\usepackage{booktabs}
\newcommand\abs[1]{\left|#1\right|}

\renewcommand{\baselinestretch}{1.05}

% Table of contents for each chapter

\usepackage[nottoc, notlof, notlot]{tocbibind}
\usepackage{minitoc}
\setcounter{minitocdepth}{2}
\mtcindent=15pt
% Use \minitoc where to put a table of contents

\usepackage{aecompl}

% Glossary / list of abbreviations

\usepackage[intoc]{nomencl}
\renewcommand{\nomname}{List of Abbreviations}

\makenomenclature

% My pdf code

\usepackage{ifpdf}

\ifpdf
  %\usepackage[pdftex]{graphicx}
  \DeclareGraphicsExtensions{.jpg}
  \usepackage[a4paper,pagebackref,hyperindex=true]{hyperref}
\else
  \usepackage{graphicx}
  \DeclareGraphicsExtensions{.ps,.eps}
  \usepackage[a4paper,dvipdfm,pagebackref,hyperindex=true]{hyperref}
\fi

\graphicspath{{.}{images/}}

% nicer backref links
\renewcommand*{\backref}[1]{}
\renewcommand*{\backrefalt}[4]{%
\ifcase #1 %
(Not cited.)%
\or
(Cited on page~#2.)%
\else
(Cited on pages~#2.)%
\fi}
\renewcommand*{\backrefsep}{, }
\renewcommand*{\backreftwosep}{ and~}
\renewcommand*{\backreflastsep}{ and~}

% Links in pdf
\usepackage{color}
\definecolor{linkcol}{rgb}{0,0,0.4} 
\definecolor{citecol}{rgb}{0.5,0,0} 
\definecolor{gray}{rgb}{0.4,0.4,0.4}
\definecolor{darkblue}{rgb}{0.0,0.0,0.6}
\definecolor{cyan}{rgb}{0.0,0.6,0.6}

\lstset{
  basicstyle=\ttfamily,
  columns=fullflexible,
  showstringspaces=false,
  commentstyle=\color{gray}\upshape
}

\lstdefinelanguage{XML}
{
  morestring=[b]",
  morestring=[s]{>}{<},
  morecomment=[s]{<?}{?>},
  stringstyle=\color{black},
  identifierstyle=\color{darkblue},
  keywordstyle=\color{cyan},
  morekeywords={annotation, folder, filename, path, source, size, width, height, depth, segmented, object, name, pose, truncated, difficult, bndbox, xmin, ymin, xmax, ymax, database}% list your attributes here
}

% Change this to change the informations included in the pdf file

% See hyperref documentation for information on those parameters

\hypersetup
{
bookmarksopen=true,
pdftitle="Design and Use of Anatomical Atlases for Radiotherapy",
pdfauthor="Olivier COMMOWICK", 
pdfsubject="Creation of atlases and atlas based segmentation", %subject of the document
%pdftoolbar=false, % toolbar hidden
pdfmenubar=true, %menubar shown
pdfhighlight=/O, %effect of clicking on a link
colorlinks=true, %couleurs sur les liens hypertextes
pdfpagemode=None, %aucun mode de page
pdfpagelayout=SinglePage, %ouverture en simple page
pdffitwindow=true, %pages ouvertes entierement dans toute la fenetre
linkcolor=linkcol, %couleur des liens hypertextes internes
citecolor=citecol, %couleur des liens pour les citations
urlcolor=linkcol %couleur des liens pour les url
}

% definitions.
% -------------------

\setcounter{secnumdepth}{3}
\setcounter{tocdepth}{2}

% Some useful commands and shortcut for maths:  partial derivative and stuff

\newcommand{\pd}[2]{\frac{\partial #1}{\partial #2}}
%\def\abs{\operatorname{abs}}
\def\argmax{\operatornamewithlimits{arg\,max}}
\def\argmin{\operatornamewithlimits{arg\,min}}
\def\diag{\operatorname{Diag}}
\newcommand{\eqRef}[1]{(\ref{#1})}

\usepackage{rotating}                    % Sideways of figures & tables
%\usepackage{bibunits}
%\usepackage[sectionbib]{chapterbib}          % Cross-reference package (Natural BiB)
%\usepackage{natbib}                  % Put References at the end of each chapter
                                         % Do not put 'sectionbib' option here.
                                         % Sectionbib option in 'natbib' will do.
\usepackage{fancyhdr}                    % Fancy Header and Footer

% \usepackage{txfonts}                     % Public Times New Roman text & math font
  
%%% Fancy Header %%%%%%%%%%%%%%%%%%%%%%%%%%%%%%%%%%%%%%%%%%%%%%%%%%%%%%%%%%%%%%%%%%
% Fancy Header Style Options

\pagestyle{fancy}                       % Sets fancy header and footer
\fancyfoot{}                            % Delete current footer settings

%\renewcommand{\chaptermark}[1]{         % Lower Case Chapter marker style
%  \markboth{\chaptername\ \thechapter.\ #1}}{}} %

%\renewcommand{\sectionmark}[1]{         % Lower case Section marker style
%  \markright{\thesection.\ #1}}         %

\fancyhead[LE,RO]{\bfseries\thepage}    % Page number (boldface) in left on even
% pages and right on odd pages
\fancyhead[RE]{\bfseries\nouppercase{\leftmark}}      % Chapter in the right on even pages
\fancyhead[LO]{\bfseries\nouppercase{\rightmark}}     % Section in the left on odd pages

\let\headruleORIG\headrule
\renewcommand{\headrule}{\color{black} \headruleORIG}
\renewcommand{\headrulewidth}{1.0pt}
\usepackage{colortbl}
\arrayrulecolor{black}

\fancypagestyle{plain}{
  \fancyhead{}
  \fancyfoot{}
  \renewcommand{\headrulewidth}{0pt}
}

\usepackage{algorithm}
\usepackage[noend]{algorithmic}
\usepackage[algo2e,ruled,lined,boxed,linesnumbered]{algorithm2e}

%%% Clear Header %%%%%%%%%%%%%%%%%%%%%%%%%%%%%%%%%%%%%%%%%%%%%%%%%%%%%%%%%%%%%%%%%%
% Clear Header Style on the Last Empty Odd pages
\makeatletter

\def\cleardoublepage{\clearpage\if@twoside \ifodd\c@page\else%
  \hbox{}%
  \thispagestyle{empty}%              % Empty header styles
  \newpage%
  \if@twocolumn\hbox{}\newpage\fi\fi\fi}

\makeatother
 
%%%%%%%%%%%%%%%%%%%%%%%%%%%%%%%%%%%%%%%%%%%%%%%%%%%%%%%%%%%%%%%%%%%%%%%%%%%%%%% 
% Prints your review date and 'Draft Version' (From Josullvn, CS, CMU)
\newcommand{\reviewtimetoday}[2]{\special{!userdict begin
    /bop-hook{gsave 20 710 translate 45 rotate 0.8 setgray
      /Times-Roman findfont 12 scalefont setfont 0 0   moveto (#1) show
      0 -12 moveto (#2) show grestore}def end}}
% You can turn on or off this option.
% \reviewtimetoday{\today}{Draft Version}
%%%%%%%%%%%%%%%%%%%%%%%%%%%%%%%%%%%%%%%%%%%%%%%%%%%%%%%%%%%%%%%%%%%%%%%%%%%%%%% 

\newenvironment{maxime}[1]
{
\vspace*{0cm}
\hfill
\begin{minipage}{0.5\textwidth}%
%\rule[0.5ex]{\textwidth}{0.1mm}\\%
\hrulefill $\:$ {\bf #1}\\
%\vspace*{-0.25cm}
\it 
}%
{%

\hrulefill
\vspace*{0.5cm}%
\end{minipage}
}

\let\minitocORIG\minitoc
\renewcommand{\minitoc}{\minitocORIG \vspace{1.5em}}

\usepackage{multirow}
\usepackage{slashbox}

\newenvironment{bulletList}%
{ \begin{list}%
	{$\bullet$}%
	{\setlength{\labelwidth}{25pt}%
	 \setlength{\leftmargin}{30pt}%
	 \setlength{\itemsep}{\parsep}}}%
{ \end{list} }

\newtheorem{definition}{Définition}
\renewcommand{\epsilon}{\varepsilon}

% centered page environment

\newenvironment{vcenterpage}
{\newpage\vspace*{\fill}\thispagestyle{empty}\renewcommand{\headrulewidth}{0pt}}
{\vspace*{\fill}}

\newcommand*{\tabbox}[2][t]{%
    \vspace{0pt}\parbox[#1][3.7\baselineskip]{3cm}{\strut#2\strut}}
\newtheorem{theorem}{Theorem}

\begin{document}

\begin{titlepage}
\thispagestyle{empty}
%Border
\begin{tikzpicture}[remember picture, overlay]
  \draw[line width = 3pt] ($(current page.north west) + (2cm,-1.5cm)$) rectangle ($(current page.south east) + (-1cm,1.5cm)$);
\end{tikzpicture}
\begin{tikzpicture}[remember picture, overlay]
  \draw[line width = 1pt] ($(current page.north west) + (1.9cm,-1.4cm)$) rectangle ($(current page.south east) + (-0.9cm,1.4cm)$);
\end{tikzpicture}
\vspace{-2cm}
\begin{center}
\large 
	\bfseries{HO CHI MINH UNIVERSITY OF TECHNOLOGY} \\ [0.3cm]
	\bfseries{COMPUTER SCIENCE AND COMPUTER ENGINEERING DEPARTMENT} \\
\end{center}

\vspace{0.4cm}
\begin{center}
\includegraphics[scale=0.35]{hcmut.png}\\[1cm]
\end{center}
\vspace{-0.75cm}
\begin{center}
\large 
	\bfseries BACHELOR DISSERTATION \\
\end{center}
%\rule{\textwidth}{1pt}
\vspace{-1.25cm}
\begin{center}
\Large
	\begin{tabular}{@{}c}
		\bfseries{A SYSTEM TO COLLECT AND ANALYZE IMAGE}\\ 
		\bfseries{AND VIDEO DATA USING CROWDSOURCING} \\		
		\bfseries{MODEL AND DEEP LEARNING TECHNIQUES} \\
		\bfseries{ FOR SOCIAL SECURITY} \\[0.5cm]
	\end{tabular}
\end{center}
%\rule{\textwidth}{1pt}\\[1cm]
	
\hspace{4.5cm}	
\begin{minipage}[t]{0.7\linewidth}
\large
	\textbf{HỘI ĐỒNG: HỆ THỐNG THÔNG TIN}\\ [0.5cm]
	\textbf{GVHD: PSG. TS. Đặng Trần khánh}\\ [0.5cm]
	\textbf{GVPB: TS. ...}\\
	\vspace{-0.7cm}
	\begin{center}
	\textbf{---o0o---}
	\end{center}
	\textbf{SVTH 1: Đinh Duy Kha ()}\\ [0.5cm]
	\textbf{SVTH 2: Võ Tuấn Kiệt (1450241)}\\[0.5cm]
	\textbf{SVTH 3: Phạm Hồng Thái ()}\\[0.5cm]
	\textbf{SVTH 4: Vũ Minh Trí ()}\\[0.5cm]
	\textbf{SVTH 5: Nguyễn Khánh Nam ()}\\[0.5cm]
\end{minipage}

\vfill
\centerline{\large{TP. HỒ CHÍ MINH, THÁNG 06 NĂM 2018}}
\end{titlepage}

\dominitoc

\pagenumbering{roman}

\cleardoublepage

\chapter*{LỜI CAM ĐOAN}
Chúng tôi xin cam đoan đề tài luận văn \textit{Đánh giá năng suất cây ăn trái bằng kỹ thuật học sâu} là công trình nghiên cứu khoa học độc lập của chúng tôi, dưới sự hướng dẫn của thầy \textbf{TS. Dương Ngọc Hiếu}. Kết quả có được do chúng tôi tự tìm hiểu, phân tích một cách trung thực và khách quan dựa trên thực tiễn của đề tài.
\\
\\
Các tài liệu, số liệu trong bài có nguồn gốc rõ ràng, được trích dẫn theo nguyên tắc khoa học và đã được công bố theo đúng quy định. Các kết quả trong đề tài nghiên cứu này chưa từng được công bố trong bất kì nghiên cứu nào khác.


\begin{flushright}
Tp. Hồ Chí Minh, ngày 19 tháng 06 năm 2018
\end{flushright}
\hspace{10cm} Tác giả luận văn \\
\vspace{3cm}





\hfill Lương Gia Kiện \hspace{2cm} Trần Ngọc Đoan Thư

\cleardoublepage

\chapter*{LỜI CẢM ƠN}
Chúng tôi xin gửi lời cảm ơn sâu sắc đến \textbf{TS. Dương Ngọc Hiếu}, thầy đã tận tình hướng dẫn và đưa ra những góp ý hết sức quý báu từ những ngày đầu chúng tôi bắt đầu nghiên cứu luận văn.
\\
\\
Bên cạnh đó, trân trọng cảm ơn các thầy, cô giáo trong khoa Khoa học và Kỹ thuật Máy tính, trường Đại học Bách khoa thành phố Hồ Chí Minh, đã nhiệt tình, tận tụy chỉ dạy, truyền đạt kiến thức trong suốt bốn năm qua để tạo nền tảng cho chúng tôi có thể thực hiện được đề tài này.
\\
\\
Xin được bày tỏ lòng biết ơn chân thành đến các tác giả, đồng tác giả của những bài báo, tài liệu,... được chúng tôi dùng tham khảo trong đề tài. Cảm ơn đội ngũ phát triển thư viện Faster R-CNN, là thư viện hỗ trợ chúng tôi thực hiện mô hình nhận diện và xử lý ảnh trong đề tài này. Nghiên cứu này sẽ không thể hoàn thành nếu thiếu những sự trợ giúp đó.


\begin{flushright}
Tp. Hồ Chí Minh, ngày 19 tháng 06 năm 2018 \\
\hfill \\
Lương Gia Kiện, Trần Ngọc Đoan Thư
\end{flushright}

\cleardoublepage

\chapter*{TÓM TẮT LUẬN VĂN}
Ngày nay, việc áp dụng khoa học kĩ thuật vào các lĩnh vực trong cuộc sống đang là vấn đề thường xuyên được đề cập tới trong xã hội. Các kĩ thuật mới, nghiên cứu mới được áp dụng các ngành nghề không những giúp tăng cường hiệu quả sản xuất mà còn giúp con người tiết kiệm thời gian sức lực trong công việc. Đối với lĩnh vực nông nghiệp, tự động hóa đã và đang được áp dụng trên các khu trồng trọt ở nước ngoài. Nhờ có sự giúp đỡ của máy móc hiện đại mà người nông dân có thể làm rất nhiều việc trong thời gian ngắn.
Các nghiên cứu trong lĩnh vực thị giác máy tính cũng đã và đang đóng góp rất nhiều cho nông nghiệp thế giới nói chung và nông nghiệp nước ta nói riêng. Thị giác máy tính được tích hợp vào các máy ảnh được trang bị cho các robot tự động ở trong vườn cây để góp phần phát hiện sâu bệnh, đếm số trái cây giúp cho việc lập biểu đồ năng suất. Nhờ có thị giác máy tính mà bài toán đánh giá năng suất của cây ăn trái đã có hướng giải quyết thích hợp.
~\\

Mạng nơ-ron tích chập (CNN) là phương pháp được nhiều nghiên cứu chỉ ra để giải quyết bài toán phân loại và nhận diện vật thể. Mô hình này có nhiều ưu điểm nổi trội, giúp trích xuất đặc trưng của đối tượng dễ dàng hơn và có khả năng biến đổi khá linh hoạt để giải quyết các bài toán về nhận diện vật thể. Do đó, nhóm đã tìm hiểu và quyết định sử dụng mô hình thuật toán Faster R-CNN để giải quyết cho bài toán đánh giá năng suất cây ăn trái bằng kĩ thuật học sâu mà điển hình là ở bước nhận diện trái cây. Dựa trên những tìm hiểu có được, nhóm tiến hành chạy giải thuật này với dữ liệu hình ảnh trái bưởi thực tế trên máy tính có sử dụng GPU và đồng thời đánh giá kết quả đạt được và đề xuất cách để cải thiện độ chính xác.

\cleardoublepage

%%%%%%%%%%%%%%%%%%% mục lục %%%%%%%%%%%%%%%%%%%
\renewcommand{\contentsname}{MỤC LỤC}
\renewcommand{\listfigurename}{MỤC LỤC HÌNH}
\renewcommand{\listtablename}{MỤC LỤC BẢNG}
\renewcommand{\figurename}{Hình}
\renewcommand{\tablename}{Bảng}

\newcommand{\footcaption}[1]{\caption[#1]{#1\footnotemark.}}

\tableofcontents
\listoffigures 
\listoftables

\cleardoublepage

\chapter*{DANH MỤC THUẬT NGỮ}

{\renewcommand{\arraystretch}{1.5}
\begin{table}[H]
    \begin{tabular}{p{4cm}  p{9cm}}    
     DNN & Deep learning Neural Network \\
     ANN & Artificial Neural Network \\
     CNN & Convolutional Neural Network \\
     R-CNN & Regional Convolutional Neural Network \\
     RoI & Region of Interest \\
     ML  & Machine Learning\\
     DL & Deep Learning \\	
     SVM & Support Vector Machine\\     
     RoI & Region of Interest \\
     RPN & Region Proposal Network \\
     BBox & Bounding Box \\
     GD & Gradient Descent\\
     NMS & Non-Max Suppression \\
	\end{tabular}   
\end{table}


\mainmatter

\chapter{INTRODUCTION}
\label{introduction}
\section{Motivation}
Along with the development of technology, data is generated with a tremendous volume. According to the statistic, Facebook users upload 250 billion photos, and 350 million new images each day \footnote{Source: \url{https://www.businessinsider.com/facebook-350-million-photos-each-day-2013-9}}. In 2016, 47 \% of Vietnamese population have access to the Internet (World bank). This explosion of data enable the opportunity to analyze and extract valuable information.

Neural network is one of the most effective technique to derive information from data. Neural networks successfully thrives in this age thanks to three factors: increase dataset's size, increase of computational resource and advancement of algorithms.

Those reasons 
 
\section{Thesis statement}
\subsection{Goal}
The goal of this thesis is to accomplish to the following tasks:
\begin{itemize}
	\item Create an interface to interact with users. In the scope of this thesis, users of the system are inhabitants of a neighbor
	\item Create data collection points with proposing crowd-sourcing methods. Collected data can be in form of images and videos form contributors such as residents of neighbors, security camera.
	\item Analyze collected data using methods which will be proposed in the following chapters, and come up with meaningful information such as warnings or notifications to users of the system.
\end{itemize} 

The goal of this thesis is to build a system to collect and analyze image/video data using crowd-sourcing model and deep learning techniques for social security. The system require these features:
\begin{itemize}
	\item Using crowd-sourcing technique to collect image/video from users.
	\item Collected data are saved into a database.
	\item Extract valuable information from collected data using deep learning.
	\item Extracted information is also saved to database and used for security matter
\end{itemize} 
The main task of the project are:
\begin{itemize}
	\item Create an user interface to interact with users. The interface allows users to upload image/video and provide as much as information about the data to the system.
	\item Design a database which is able to save data, information about data, extracted information from data,...
	\item Design a deep learning model to extract information from data.
	\item Using extracted information to notify users about security problem.
\end{itemize} 

\subsection{Stages}
Nhận diện vật thể không phải là một chủ đề mới trong lĩnh vực thị giác máy tính. Mô hình mạng nơ-ron tích chập (Convolutional Neural Network) đã mở ra nhiều hướng đi cho bài toán học có giám sát (Supervised Learning) và đã chứng minh sức mạnh của nó đối với dữ liệu huấn luyện lớn. Điểm mạnh của mô hình mạng tích chập là có thể được sử dụng để để huấn luyện một bộ xử lí “end to end”, nghĩa là nó có thể nhận dữ liệu đầu vào dưới dạng gốc như là một bức ảnh và đưa ra được kết quả phân loại của bức ảnh đó. Mặc dù tốt như vậy nhưng nó cũng tồn tại một điểm yếu lớn, đó chính là cần một lượng dữ liệu đầu vào lớn để được vào huấn luyện cho bộ học và việc gán nhãn cho dữ liệu huấn luyện rất mất thời gian và công sức nếu muốn có được một tập dữ liệu đa dạng và chính xác. Vì vậy việc chuẩn bị, sàn lọc, xử lí dữ liệu đầu vào là rất cần thiết đối với một giải thuật dựa trên mạng nơ-ron tích chập.

Nhận diện trái cây không phải là một chủ đề mới, đã có rất nhiều nghiên cứu liên quan về bài toán này \cite{bargoti2017image} \cite{sa2016deepfruits}. Một trong những phương pháp nhận diện trái cây đã được đề xuất trước đây trong bài báo của Cohen \cite{cohen2010estimation}. Hình ảnh họ sử dụng được chụp bởi camera theo chuẩn màu CCD. Đầu tiên, họ dùng một bộ phân loại K-nearest-neighbors (KNN) để xác định xem những điểm ảnh (pixel) nào là "táo" và những điểm ảnh nào là "không-phải-táo", những vật thể che mất quả táo như cành, lá được đánh dấu lại để loại bỏ ra khỏi quá trình huấn luyện. Sau đó đánh dấu bề mặt  của quả táo bằng cách cho nhận diện những vùng mà họ gọi là "seed area". Seed area là tập hợp những điểm ảnh có khả năng cao là táo. Sau đó những vùng seed area này được mở rộng ra để liên kết những vùng ảnh quá sáng hoặc quá tối nằm giữa hai seed area, từ đó tạo thành một vùng seed area hoàn chỉnh. Cuối cùng, họ phân tích các hình dạng của seed area và khoanh vùng được quả táo từ những đường nét của mỗi seed area. Tuy nhiên phương pháp này của họ vẫn chưa hiệu quả khi quả có quá nhiều hình dạng và màu khác nhau, nó cờn bị ảnh hưởng mạnh bởi độ sáng, bóng râm.

Nhiều nghiên cứu đều cho thấy vấn đề của quá trình nhận diện vật thể đó chính là công tác phân đoạn (segmentation), phải phân biệt rõ giữa vùng có vật thể và vùng nền chứa vật thể. Nhóm của Yamamoto \cite{yamamoto2014plant} đã sử dụng phương pháp phân đoạn hình ảnh dựa trên màu sắc để áp dụng cho học các đặc trưng trên ảnh. Mạng nơ-ron tích chập cũng có ưu điểm rõ ràng, đó chính là không cần phải trích xuất đặc trưng một cách thủ công. Các đặc trưng được trích xuất thông qua mạng tích chập được sử dụng vào quá trình nhận diện ảnh, có thể phân tích hình ảnh để lấy những đặc trưng low-level để giảm kích thước không gian nhận diện nhằm xác định vùng quan tâm (Region of Interests - RoIs) đồng thời khai thác đặc trưng high-level áp dụng vào quá trình phân loại.

Mô hình mạng R-CNN (Region based Convolutional Neural Network) cũng được đề ra nhằm giải quyết bài toán nhận diện. Những vùng quan tâm (RoIs) được tạo ra từ giải thuật Selective Search, sau đó được đưa qua mạng tích chập để được phân loại, đồng thời nó còn được sử dụng để tính toán hồi quy tìm ra bounding box. Faster R-CNN là mô hình tích hợp giữa các công việc là tìm vùng quan tâm, phân loại vật thể và tính toán hồi quy để tìm ra bounding box. Nhờ như vậy nên việc nhận diện vật thể trở nên nhanh hơn và hiệu quả hơn. Không những vậy, mô hình Faster R-CNN cho thấy một kết quả khả quan có thể áp dụng vào thực tế cho bài toán nhận diện trái cây \cite{bargoti2017deep}.

Tuy nhiên, đa số các nghiên cứu hiện nay đều hướng tới kết quả nhận diện được nhiều loại trái cây, đa phần là những loại trái cây ở khu vực nước ngoài, vì vậy việc áp dụng với các loại trái cây ở Việt Nam cũng rất cần được quan tâm. Do đó nhóm sẽ tập trung vào nhận diện, đánh giá những trái có tính chất như vậy, tiêu biểu là trái bưởi. Bài báo cáo này trình bày giai đoạn đầu tiên trong việc Xác định năng suất cây trồng bằng mạng học sâu, đó chính là nhận diện vị trí và phân loại trái cây. Để đạt được kết quả tốt nhất, nhóm thống nhất chọn mô hình mạng tích chập để dễ dành trích xuất đặc trưng từ  dữ liệu hình ảnh và framework Faster R-CNN, đã được cải tiến để giảm thiểu tối đa chi phí với một mạng rất sâu.

Ở phần sau, nhóm sẽ trình bày những nội dung sau:
\begin{itemize}
	\item Chương 2: Cơ sở lí thuyết
	\item Chương 3: Giải pháp đề xướng
	\item Chương 4: Ứng dụng Faster R-CNN vào trong nhận diện trái bưởi
	\item Chương 5: Kết luận và hướng phát triển, trình bày thêm về bộ phân loại áp dụng vào mô hình
\end{itemize} 
\section{Scientific and Practical contributions}
\subsection{Scientific contributions}
Develop a system to collect data and allow user to interact with it through posting and notifications. \\
By applying Deep Learning and Computer Vision theories and by utilizing the Keras framework with a Tensorflow backend, some video classification techniques can be re-implemented and benchmarked. Subsequently, the team is able to apply those techniques to a real-world problem which is finding suspicious activities in videos \\ %FIX.
In addition, those classifying models are deployed with performance in mind using NGINX ,Gunicorn and Flask framework.
\subsection{Practical contributions}
The thesis proposes a system to improve security of neighborhoods. Such system can be beneficial to the residents as it assure their safety.
\section{Thesis scope}
This thesis focus on building the system at a scope a neighborhood. 
Đóng góp một phương án để cải thiện độ chính xác, tăng số trái nhận diện được trong một ảnh, bằng cách kết hợp với một bộ phân loại nhằm phân biệt trái cây với nền, như vậy số vật thể phân loại sai sẽ giảm, đồng thời tỉ lệ nhận diện đúng số lượng trái sẽ tăng lên.


\section{Report overview}


\cleardoublepage

\chapter{BACKGROUND KNOWLEDGE}
\label{chap:background}
\paragraph{Chương 2} trình bày các kiến thức nền tảng cần thiết phục vụ cho quá trình thực hiện đề tài, bao gồm các nội dung:

\begin{itemize}
\item Giới thiệu về mạng học sâu, một phạm trù của Machine Learning

\item Định nghĩa, cấu trúc, cách hoạt động của mạng neural nhân tạo

\item Mở rộng của mạng neural nhân tạo - mạng neural tích chập - cấu trúc, các lớp, cơ chế tính toán và ứng dụng của nó trong việc xử lí dữ liệu có dạng hình ảnh

\item Các thuật toán nhận diện hình ảnh sử dụng mô hình neural tích chập, gồm: R-CNN, Fast R-CNN và Faster R-CNN
\end{itemize}

\section{MVC model}
\section{Database}
\section{Face recognition}
\section{Action recognition}
\section{Deep neural network}
\section{Cloud computing}


\cleardoublepage

\chapter{PROPOSED SOLUTION}
\label{chap:solution}

This chapter describes a proposed solution which is a considerable system consists of two subsystems: Crowd-sourcing system and  analysis system. In \textit{System overview} subsection, system is briefly described altogether. In two last subsections, structure and details of two subsystems are defined comprehensively.

\section{System overview}
\subsection{Requirement analysis}
To be able to utilize data to serve a security purpose, the most obvious method would be to analyze visual data such as photos and videos.  In recent years, advancements in Artificial Intelligence, especially in Neural Networks, significantly enhanced the development of computer vision field. Project Adam from Microsoft() The use of Deep learning comes along with needs for proper datasets. There are really few datasets that are suitable for the environment of Vietnam. 
The traditional way to collect data would be to set up an array of cameras, then
Crowd-sourcing is one of the effective methods of data collection. The idea behind crowd-sourcing is to build datasets with the help of a large group of people. An example of this kind of model is Wikipedia. Wikipedia is an enormous web-based, collaborative encyclopedia which has over 100,000 volunteers contributing new information to the system daily. The success of Wikipedia proves that people gladly contribute to a system without profit if it brings a greater good. 
But how can people be encouraged to contribute their knowledge and information? Through interaction with others had been proven successful in increasing engagement. Social media has been a popular trend among Vietnamese people. Therefore, this thesis proposes building a social media platform for security. This social media serves as a point of interaction with users, as well as allow people to contribute their knowledge to the system.
Figure \ref{chap3:system_overview_basic} shows an concise overview of how the system operate. Users interact with the \textbf{Crowd-sourcing system} through \textbf{User Interfaces}. Input of users can come in the form of images, videos or label contribution. Inputs of users are stored in the database. The system also be responsible for obtaining appropriate contents from database to display to users through user interfaces.
\begin{center}
    \begin{figure}[H]
    \centering
    \includegraphics[width=1\columnwidth]{images/chap3/system_overview_basic.png}
    \footcaption{An overview of the system}
    \label{chap3:system_overview_basic}
    \end{figure}
\end{center}
Whenever there is a video to analyze,the server will send it to the \textbf{Video classifier module}, then get the results back. Results returned from video classifier module are classified activity from the video and frames that contain faces in them. \textbf{Face Recognition module} does the job of identifying people in images. Results of both \textbf{Video classifier module} and \textbf{Face Recognition module} are used to determine security threats each video or image shows.
\section{Crowd-sourcing system}
\subsection{Database}
The project database divides into two different parts: \textbf{Cloud storage} and a \textbf{NoSQL database}. Why it requires such a complex system? The primary target is to reduce website loading time. Let take an example, if files are stored directly on the crowd-sourcing server. When many users request a file simultaneously, the server with limited bandwidth will cause delay. “53\% of mobile site visitors leave a page that takes longer than three seconds to load” – \href{https://think.storage.googleapis.com/docs/mobile-page-speed-new-industry-benchmarks.pdf}{Google}. What if these files stored on cloud storage? The client will be served by the cloud storage provider, which has higher availability.
\section{Analysis system}
The project requires two systems for analysis: A face recognition system and a video classifier system. The facial recognition system takes pictures of human faces as input and return their identification. The video classifier system analyzes videos to find out actions in them. The remaining of this section describes in detail about the two analysis system.
\subsection{Face recognition module}
\subsection{Video classifier module}


\cleardoublepage

\chapter{SYSTEM DESIGN AND IMPLEMENTATION}
\label{chap:caseFarming}

This chapter describes the implementation of the system which consists of two subsystems: A social media website and an analysis server. In \textit{Requirement analysis and System overview} subsection, everything is briefly described as a whole. In three last sections, structure and details of two subsystems and the database are defined more comprehensively.

\section{Requirement analysis and system overview}
\subsection{Requirement analysis}

\subsection{System overview}
Figure \ref{chap3:system_overview_basic} shows a concise overview of how the system operates. Users interact with the website through front-end. The input of users can come in the form of images, videos or label contribution. Back-end receives input and stores in the database. Concurrently, inputs are sent to analysis server . Outputs from analysis server are sent back to back-end and then stored in database. The back-end is also responsible for obtaining appropriate contents from the database to display to users through front-end.

\begin{center}
    \begin{figure}[H]
    \centering
    \includegraphics[width=1\columnwidth]{images/chap3/system_overview_basic.png}
    \caption{An overview of the system}
    \label{chap3:system_overview_basic}
    \end{figure}
\end{center}

\section{Social media website}

\subsection{Database}
The project database divides into two different parts: \textbf{Cloud storage} and a \textbf{NoSQL database}. The primary target is to reduce website loading time. Let take an example, if files are stored directly on the crowd-sourcing server. When many users request a file simultaneously, the server with limited bandwidth will cause delay. “53\% of mobile site visitors leave a page that takes longer than three seconds to load” – \href{https://think.storage.googleapis.com/docs/mobile-page-speed-new-industry-benchmarks.pdf}{Google}. If these files stored on cloud storage, the client will be served by the cloud storage provider, which has higher availability.
\section{Analysis server}
The project requires two systems for analysis: A face recognition system and a video classifier system. The facial recognition system takes pictures of human faces as input and returns their identification. The video classifier system analyzes videos to find out actions in them. The remaining of this section describes in detail about the two analysis system.
\subsection{Face recognition module}
\subsection{Video classifier module}
Video classifying is not a new task in the field of Deep learning.
	
\section{Technologies}
\subsection{Node.js and npm}
Node.js (Node) is a JavaScript runtime built on Chrome's V8 JavaScript engine (cite). Node supports executing JavaScript on server-side. Ryan Dahl was created Node.js in 2009. Node.js Foundation is in charge of the development of Node.js. After nine years since its release, the latest LTS version of Node.js is 10.14.2 (includes npm 6.4.1). “Node.js operates on a single thread, using non-blocking I/O calls, allowing it to support tens of thousands (cite) of concurrent connections held in the event loop.” Although JavaScript is single-threaded, thanks to the Event loop, Node.js can implement asynchronous I/O operations. The Event loop will transfer operations to the system kernel. “Since most modern kernels are multi-threaded, they can handle multiple operations executing in the background. When one of these operations completes, the kernel tells Node.js so that the appropriate callback may be added to the poll queue to executed eventually”. By utilizes non-blocking I/O, Node.js skips the waiting time for I/O calls, which is much higher than processing time.
\section{Implementation}

\cleardoublepage

\chapter{EVALUATION}
This chapter presents the process of evaluating the effectiveness of the whole system and its features. At the same time, self-assessment of the work done by the group, from which the development direction of the topic
\section{Social media website}
\subsection{Evaluation metric}
\subsection{Result}
\section{Video classifier}
\subsection{Evaluation metric}
\subsection{Result}




\cleardoublepage

\chapter{CONCLUSION AND FURTHER DEVELOPMENT}
\label{chap:relatedwork}

In Chapter 6, we describe a few recent works relating to river runoff prediction and boiler efficiency optimization.

\minitoc

\section{Accomplishment}
\section{Limitation}
\section{Further development}
\section{Instruction}
\subsection{Verifying account}
1. Login with your account or register new account. 
\begin{center}
    \begin{figure}[H]
    \centering
    \includegraphics[width=1\columnwidth]{images/chap6/instruction1.png}
    \footcaption{Homepage}
    \label{}
    \end{figure}
\end{center}
2. Unverified account is unable to use most of the features, verify your account by clicking your name to go to your profile page
\begin{center}
    \begin{figure}[H]
    \centering
    \includegraphics[width=1\columnwidth]{images/chap6/instruction2.png}
    \end{figure}
\end{center}
3. Enter your phone number to verify
\begin{center}
    \begin{figure}[H]
    \centering
    \includegraphics[width=1\columnwidth]{images/chap6/instruction3.png}
    \end{figure}
\end{center}
4. User can choose either to receive confirmation code by WhatsApp or SMS. Enter your code to finish verifying.
\begin{center}
    \begin{figure}[H]
    \centering
    \includegraphics[width=1\columnwidth]{images/chap6/instruction4.png}
    \end{figure}
\end{center}
\subsection{Label video}
1. Go to "Label video" page
\begin{center}
    \begin{figure}[H]
    \centering
    \includegraphics[width=1\columnwidth]{images/chap6/instruction5.png}
    \end{figure}
\end{center}
2. Choose either "Suspicious" or "Not suspicious". 
\begin{center}
    \begin{figure}[H]
    \centering
    \includegraphics[width=1\columnwidth]{images/chap6/instruction6.png}
    \end{figure}
\end{center}
\subsection{Create post}
1. Go to "Create post" page
\begin{center}
    \begin{figure}[H]
    \centering
    \includegraphics[width=1\columnwidth]{images/chap6/instruction7.png}
    \end{figure}
\end{center}
2. Fill in the form to and click "Create". 
\begin{center}
    \begin{figure}[H]
    \centering
    \includegraphics[width=1\columnwidth]{images/chap6/instruction8.png}
    \footcaption{Create post form}
    \end{figure}
\end{center}
\subsection{Subscribe a location}
1. Go to profile page and choose "All locations" tab
\begin{center}
    \begin{figure}[H]
    \centering
    \includegraphics[width=1\columnwidth]{images/chap6/instruction9.png}
    \end{figure}
\end{center}
2. Choose a location to subscribe from the list. User will receive notification about suspicious behavior around subscribed location within a radius of 1 kilometer.   
\begin{center}
    \begin{figure}[H]
    \centering
    \includegraphics[width=1\columnwidth]{images/chap6/instruction10.png}
    \end{figure}
\end{center}




Hình \ref{chap2:neural_model} mô tả mô hình một neural trong mạng neural nhân tạo.
\begin{center}
    \begin{figure}[H]
    \centering
    \includegraphics[width=0.6\columnwidth]{images/chap2/neuron.png}
    \footcaption{Cấu tạo một neural thần kinh}
    \label{chap2:animal_neural}
    \end{figure}
\end{center}
\footnotetext{Source: \url{http://cs231n.github.io/neural-networks-1/}}





\cleardoublepage

\chapter{Conclusion and Perspectives}
\label{conclusion}

\section{Conclusion}

In Vietnam, agriculture is one of the major fields. It recently contributed approximately 15-20$\%$ to the national GDP. Rice exports contributed about 1.8 billion USD in 2015. Therefore, problems involving agriculture attract plenty of attention from scientists, managers, and even the government in Vietnam. However, scant significant research, especially regarding applications of computer science to hydrology and fertilizer production, have been deployed successfully into practice during the past few years. Water resources and fertilizer are the most important elements influencing the productivity of rice plants. Thus it is necessary to promote research involving water resources and fertilizer production and apply the results to practice.

Response to this practical demand, in this thesis we study artificial neural networks and related hybrid methods. Then we apply the studies to practical and urgent problems affecting Vietnamese agriculture: river runoff prediction and boiler efficiency optimization. River runoff prediction belongs to the hydrology field, whereas boiler efficiency optimization involves fertilizer production. We attempt to solve these two completely different problems not only in theory but also in practice. One of our solutions has been deployed successfully. Among several viable methods, we chose artificial neural networks as the key one because of the straightforward idea and easy deployment. We addressed some drawbacks of artificial neural networks by combining them with fuzzy systems, evolutionary algorithms (genetic algorithm), chaotic expressions, and clustering algorithms. Depending on the different objectives of sub-problems, various hybrid methods are used. Table \ref{chap7:table01} shows our hybrid methods, their corresponding applications and publications in this thesis.

{\renewcommand{\arraystretch}{1.5}
\begin{table}
  \begin{center}
    \begin{tabular}{| p{3.0 cm} | p{3.0cm} | p{3.5cm} | p{2.5cm} |}
    \hline
    Problems & Sub-Problems & Methods & Publications \\
    \hline
	\multirow{2}{3.0 cm}{Srepok runoff prediction} & Short-term prediction 
	& 	
	RFNN
	
	RFNN-KM-Euclid
	
	RFNN-KM-DTW
	
	RFNN-DB-DTW 	
	& 
	[\ref{mypub03}], [\ref{mypub07}] \\
	\cline{2-4}
	& Long-term prediction & 
	RFNN

	RFNN-GA
	
	SWAT & 
	[\ref{mypub02}], [\ref{mypub05}], [\ref{mypub06}] \\
	\hline
	\multirow{2}{3.0 cm}{Boiler efficiency optimization} & Boiler efficiency simulation 
	& 
	RFNN
	& [\ref{mypub01}] \\
	\cline{2-4}
	& MSA real time boiler efficiency forecasting  
	& 
	RFNN
	
	SE-RFNN
	
	RTRL-RFNN 
	& 
	[\ref{mypub04}] \\
	\hline
	\end{tabular}
    \caption{Statistic of proposed methods, corresponding problems and publications}
     \label{chap7:table01} 
  \end{center} 
\end{table}

\paragraph{River Runoff Prediction.}

We divide the task of river runoff prediction into two cases: short-term and long-term. For short-term prediction, the experimental results show that a mixture of RFNNs that utilizes DBSCAN and DTW for clustering and distance-measuring, respectively, is the best combination. The performance of RFNN-DB-DTW encourages further practical deployment. For short-term prediction, SWAT, RFNN and a hybrid of RFNN and Genetic Algorithm are used. Based on the experimental results, RFNN and RFNN-GA clearly outperformed SWAT; among the three methods, RFNN-GA is the best method. Like RFNN-DB-DTW, RFNN-GA can definitely be applied for practical deployment. 

In Vietnam, there are many large rivers and some of them play a central role in people's livelihoods and in production, e.g., the MeKong River in southern Vietnam, the Srepok River in the Central Highland of Vietnam, and the Hong River in northern Vietnam. Due to the sloping terrain, the Srepok runoff has large margins; it is very high in the rainy season and very low (almost out of water) in the sunny season. Moreover, there are several natural abnormalities that often occur in the Srepok basin, e.g., storms, droughts, landslides. Therefore, the Srepok runoff contains several anomalies. In contrast, the MeKong basin and the Hong basin are plains (flat terrain) and have few natural abnormalities. Thus Srepok runoff prediction is more difficult than MeKong or Hong runoff prediction. 

However, the experimental results of the Srepok runoff prediction indicate that we can solve this problem. In fact, the proposed methods to predict the Srepok runoff can be applied to other rivers such as the MeKong River or the Hong River.    

\paragraph{Boiler Efficiency Optimization.} 

We used RFNN and some hybrids of RFFN to build a soft sensor called BEO for Phu My Fertilizer Plant. RFNN and associated methods such as RFNN-SE and RTRL-RFNN are proposed to implement two important modules of the soft sensor: Boiler Efficiency Simulation and Multi-Step-Ahead Real-Time Boiler Efficiency Forecasting. We deployed the soft sensor without the MSA Real-Time Boiler Efficiency Forecasting Module in 2013-2014; the soft sensor brought a benefit to Phu My Fertilizer Plant of approximately 55,000 USD per year. The experimental results of the MSA Real-Time Boiler Efficiency Forecasting Module were remarkable, and this module will be plugged into the new version of BEO. However, it is necessary to verify this new benefit of BEO by deploying it at Phu My Fertilizer Plant. Because of the strict policy of the plant, we are waiting for a suitable time to deploy and assess BEO. 

\section{Perspectives}

\paragraph{Climate Change and River Runoff Prediction.}

River runoff prediction does not significant benefit if it is stand-alone. In \cite{swatref01}, we proposed an information system for integrating, storing, and analyzing many kinds of data involving climate change in the Srepok basin, e.g., climate data, water resources, soil resources, etc. The data schema of the information system called SRClim is illustrated in Figure \ref{chap07:fig01}. SRClim was created in 2013 and has been developing ever since. The objective for SRClim is that it must integrate all necessary data involving climate change of the Srepok basin. Furthermore, SRClim must ensure high levels of the data's availability, security, consistency, analysis, visualization, etc. Therefore, we will plug the function of river runoff prediction into SRClim. 

\begin{figure}[H]
  \centering
  \includegraphics[width=1.0\textwidth]{images/Chap7/SRClim.pdf}
  \caption{The data schema of SRClim}
  \label{chap07:fig01}    
\end{figure}

Moreover, SRClim will be extended to connect with all hydrology stations and climate stations via a virtual private network (VPN) that will permit SRClim to collect data automatically and in real time from these stations. In this context, the function of multi-step-ahead real-time river runoff forecasting is also necessary for SCRlim; we can utilize SE-RFNN or RTRL-RFNN to implement the function.

In addition, we will verify our proposed methods for other rivers such as the MeKong River or the Hong River. We collected Hong runoff data from 1960 to 2006. Furthermore, we will also research other advanced methods such as deep learning. Utilizing deep learning, particularly deep belief networks, is appropriate for the task of river runoff prediction (the results were published in [\ref{mypub07}]). In further research, we will conduct more experiments with many settings of the deep learning model, and with many different datasets such as those of the Srepok runoff and the Hong runoff.

\paragraph{Boiler Efficiency Optimization.}

As mentioned above, it is necessary to verify the improved benefit of the new version of BEO by deploying it at Phu My Fertilizer Plant. Because of the strict policy of the plant, we are waiting for a suitable time to deploy and assess BEO. Based on theoretical analysis, we have a strong chance of success. In addition, we need to develop the function of anomaly detection that detects and removes noise (anomalies). It is an important function that we first focused on at the beginning of the project. Due to a dearth of computer science knowledge and experience with boilers, the function was not deployed successfully. Although the anomalies rarely appear, they do impact the overall performance of the soft sensor.  

\section{Publications}

In conclusion, this thesis addressed some practical and urgent problems in Vietnam by proposing some methods that improve upon artificial neural networks. The experimental results prove that our proposed methods are appropriate for tackling the problems and can be deployed in practice. The proposed methods and their experimental results have been presented and published in high-quality international conferences and journals. The publications are listed as follows.

\begin{enumerate}
\item \label{mypub01} Hieu N. Duong, Hien T. Nguyen, Vaclav Snasel and et al. \textit{Optimizing Boiler Efficiency by Data Mining Techniques: A Case Study}. In Proc. of International Conference on Information Resources Management, 2014. 
\item \label{mypub02} Hieu N. Duong, Hien T. Nguyen, Vaclav Snasel and et al. \textit{Applying Recurrent Fuzzy Neural Network to Predict the Runoff of Srepok River}. In Proc. of 13th IFIP TC8 International Conference, CISIM, pages 55-66, 2014. 
\item \label{mypub03} Hieu N. Duong, Hien T. Nguyen, Vaclav Snasel. \textit{A Hybrid Approach For Predicting River Runoff}. In Proc. of The Second Euro-China Conference on Intelligent Data Analysis and Applications, 2015. 
\item \label{mypub04} Hieu N. Duong, Hien T. Nguyen, Vaclav Snasel and et al. \textit{A Hybrid Approaches For Forecasting Real Time Multi-Step-Ahead Boiler Efficiency}. In Proc. of ACM International Conference on Ubiquitous Information Management and Communication, 2016. (was selected as one of remarkable papers to improve and submit for ETRI Journal indexed in SCIE)
\item \label{mypub05} Hieu N. Duong, Hien T. Nguyen, Vaclav Snasel and et al. \textit{Predicting Monthly River Runoff Using Recurrent Neural Fuzzy Networks and Genetic Algorithm}. In Proc. of International Conference on Information and Convergence Technology for Smart Society, 2016.
\item \label{mypub06} Hieu N. Duong, Hien Thanh Nguyen, Vaclav Snasel, Sanghyuk Lee. \textit{A comparative study of SWAT, RFNN and RFNN-GA for predicting river runoff}. Indian Journal of Science and Technology indexed in Scopus, ISI, v.9(16), 2016. (accepted)
\item \label{mypub07} Nguyen Cao Tri, Hieu N. Duong, Tran Van Hoai, Vaclav Snasel. \textit{Predicting Daily River Runoff Using Deep Belief Networks}. In Proc. of International Conference on Information and Convergence Technology for Smart Society, 2016.
\end{enumerate}



\cleardoublepage

\renewcommand\bibname{TÀI LIỆU THAM KHẢO}
\bibliographystyle{ThesisStyle}
\bibliography{Thesis}

\cleardoublepage

\chapter*{PHÂN CÔNG CÔNG VIỆC}
\begin{table}[H]
    \begin{tabular}{| p{4cm} | p{11cm} |}
    \hline
    \multirow{4}{*}{Trần Ngọc Đoan Thư} & Tìm hiểu lí thuyết về ANN, CNN \\ 
    							
    								& Thực hiện lấy mẫu ảnh, kỹ thuật augmentation để làm đa dạng tập dữ luyện ảnh \\
  
    								& Tính toán số liệu kiểm tra thu được \\
    								& Viết báo cáo phần lấy mẫu và augmentation, ANN, CNN \\
    
			 		    \hline
   \multirow{3}{*}{Lương Gia Kiện} &  Tìm hiểu R-CNN, Fast R-CNN, Faster R-CNN cấu trúc và ý tưởng mô hình \\
	         						  
			 						 & Cài đặt môi trường để thực hiện thí nghiệm, cấu hình source code, sửa lỗi, tiến hành chạy huấn luyện và kiểm tra\\
			 						 & Viết các báo cáo những phần còn lại\\
			 
			     \hline
	\end{tabular}
\end{table}


\end{document}
%\appendix
%\chapter{The Soft Sensor -  BEO}
\label{chap:appendix1}

\section{Architecture of BEO}

\begin{figure}[H]
  \centering
  \includegraphics[width=1.0\textwidth]{images/Chap5/BEOArchitecture.pdf}
  \caption{Architecture of BEO}
  \label{App:fig01}    
\end{figure}

Figure \ref{App:fig01} shows our system architecture, namely BEO - Boiler Efficiency Optimization. The system includes several complex modules. We present these modules as follows.

\subsection{Real-Time Monitoring} 

This module is an OPC (OLE for Process Control) Client collecting automatically data from OPC Server via a local network. OPC is a software interface standard that allows Windows programs to communicate with industrial hardware devices\footnote{\url{http://www.opcdatahub.com/WhatIsOPC.html}}. In every duration, the tool gets parameter's values from OPC Server and the value of duration is defined by the user, usually 60s.

\subsection{Data Pre-Processing} 

The module has two main functions: (i) Reading raw operational data from several separate historical text files or receiving real-time data from the real-time monitoring module; then analyzing the data structure and storing the data into a unit database (SQL Server DBMS); (ii) Cleaning the database to make sure it has no errors, outliers, and noisy data. Each record in the database consists of control parameters, load of the boiler, and a corresponding boiler efficiency.

\subsection{Data Clustering} 

Since the system works in real-time, the operational data is enormous. Reducing the size of the operational data is significant in speeding up the system to meet its real-time requirement. Data clustering is to group similar records into the same cluster, and derives a knowledge base consisting of the centers of those clusters. 

\subsection{Anomaly Detection}

This module is responsible for detecting anomalies of the real boiler during operating. The module is integrated in Pre-Processing data Module to remove outliers. Moreover, the module detects some abnormalities caused by unknown reasons and warns the operators to determine timerly solutions.  

\subsection{Efficiency Calculator}

The boiler efficiency can be calculated by the boiler simulator module. Typically, boiler simulator only works on M important parameters that are chosen by experts. However, each tuple of the operating data is a set of N parameters and N is usually larger than M. Several formulas can calculate the boiler efficiency from full N parameters. Unfortunately, these formulas are very complex. Two methods are used in the BEO soft sensor to calculate the boiler efficiency \cite{boi:ref_11}.

\begin{itemize}

\item \textit{Direct Method:} The boiler efficiency is the ratio of the energy obtained from steam and the energy of the fuel in the boiler.
\item \textit{Indirect Method} The boiler efficiency is the difference between energy loss and energy input.

\end{itemize}

\subsection{Boiler Efficiency Simulation}

Combustion is considered as a process of time-oriented technology, and the equation of state combustion efficiency is a complex nonlinear equation where coefficients are not fixed. It is difficult to exactly find coefficients of that nonlinear equation. As a result, we need an approximate solution. Neural networks seem to be a simple and effective approximation scheme for this kind of problem. In BEO soft sensor, the boiler with the internal reaction equations is monitored as a black box with control parameters and an corresponding output called the boiler efficiency. Therefore, we need a boiler simulator for simulating the real boiler from the operational data. The boiler simulator determines the correlation between the control parameters and the boiler efficiency. The control parameters consisting of air flow, air pressure, water flow, etc. to operate the boiler. The boiler is simulated by a multi-variable equation $y = f(x_1, x_2, ..., x_n)$, where $x_i$ is a control parameter, $y$ is a boiler efficiency, and $f(.)$ is a suggested model, e.g., neural networks. This modeling of the boiler helps us to build a boiler simulation with technological features like a real boiler. Among many advanced techniques that can approximate the real boiler, such as fuzzy systems, support vector machines, etc., RFNN is chosen due to its straightforward idea and easy deployment.

As presented in detail in Section \ref{chap:boilerefficiencyoptimization}, the Boiler Efficiency Simulation Module is implemented by RFNN with mean absolute relative error (MARE) approximately $2.04E-03$ and $2.97E-03$, in the training phase and the testing phase, respectively.

\subsection{Boiler Efficiency Optimization}

At the time of the boiler efficiency appears downtrends, this module detects in knowledge base and finds a tuple of control parameters that gives higher efficiency than current efficiency. The controllable parameters will be changed by the new values in the found tuple. After suggesting the new parameter values, the Boiler Efficiency Simulation Module predicts the boiler efficiency according to these new values. In the case of the predicting efficiency is lower than the current efficiency; the suggestion will be ignored. Conversely, this module will apply the new found parameters for the boiler with expecting an improved boiler efficiency. Process working of the module is illustrated in Figure \ref{App:fig02}, and it has several steps as follows.

\begin{itemize}

\item \textit{Step 1:} Reading data from OPC Server via the Real-Tiem Monitoring Module.
\item \textit{Step 2:} Finding some similar tuples with the current control parameters that have higher efficiency than the current and have the same load.
\item \textit{Step 3:} Checking whether the change of load is higher than a delta value. That the change is not higher a delta value indicates the load is stable and the module can adjust parameters to improve boiler efficiency. Conversely, that the change is higher a delta value indicates the load is not stable, the module ignores and continue monitoring.
\item \textit{Step 4:} In the case of stable load, if the new similar tuples can give higher efficiency that is predicted by the Boiler Efficiency Simulation Module, these tuples will be applied to improve the boiler efficiency.

\end{itemize}

\begin{figure}[H]
  \centering
  \includegraphics[width=0.5\textwidth]{images/Appendix1/BoilerOptimization.pdf}
  \caption{Real-time optimization work-flow}
  \label{App:fig02}    
\end{figure}

\subsection{Boiler Controller}

The same as the Real-Time Monitoring Module, the Boiler Controller Module is an OPC Client controlling the real boiler through OPC Server. According to some adjustments recommended by the Real-Time Boiler Efficiency Optimization Module, the Boiler Controller Module adjusts the control parameters of the real boiler to improve its efficiency.

\subsection{Multi-Step-Ahead Real-Time Forecasting}

As presented in \ref{chap:boilerefficiencyoptimization}, this module is responsible for forecasting the downtrends of boiler efficiency. In the case of down-trend appearances, this module will inform to the Real-Time Boiler Efficiency Optimization Module to proceed the adjustment of control parameters.

\section{Benefit of BEO}

We employed a statistical method called statistical inference for two samples \cite{boi:ref_14} to assess the efficiency of the BEO soft sensor to Phu My Fertilizer Plant. Note that, to calculate confidently, this method requires that the size of samples must be large so that Student's t-distribution comes close to a normal distribution. We collected the operational data with the duration of 1 sample/60s. 

Data used for assessment was collected in 17 days with two separate periods: (i) 1st period: from November 02, 2013 to November 07, 2013, the size of samples is 5892 and the boiler load distributes from 76 ton/h to 83 ton/h. (ii) 2nd period: from November 08, 2013 to November 18, 2013, the size of samples is 8834 and boiler load distributes from 72 ton/h to 84 ton/h. We defined the minimum size of samples is 30 for each boiler load. During the time of collecting data, the boiler operation was in auto-mode and the boiler load is continuous. Therefore, we must rounded many values of boiler load to get one rounded value. For example, all boiler loads from 68.5 ton/h to 69.49 ton/h are rounded to 69 ton/h. After collecting data, we employed the statistical inference method to prove that the BEO soft sensor improve the performance of power consumption. 

As mentioned in \cite{boi:ref_14}, Student's t-distribution is one of normal distribution families. In Student's t-distribution, mean of $n$ observed data is estimated as below.

\begin{equation}
\bar{x} = \frac{\sum_{i=1}^nx_i}{n},
\end{equation}

and standard deviation $\sigma$:

\begin{equation}
\sigma = \sqrt{\frac{\sum_{i=1}^n(x - \bar{x})}{n-1}}.
\end{equation}

We define some factors that are used in Tables \ref{App:table02} and \ref{App:table03}.

\begin{itemize}
\item During running the boiler with BEO, the mean of power consumption is $\bar{x}^{BEO}$.
\item During running the boiler with BEO, the standard deviation of power consumption is $\sigma_{\bar{x}}^{BEO}$.
\item During running the boiler without BEO, the mean of power consumption is $\bar{x}^{noBEO}$.
\item During running the boiler without BEO, the standard deviation of power consumption is $\sigma_{\bar{x}}^{noBEO}$.
\end{itemize}

\begin{table}[H]
\scriptsize
  \begin{center}
    \begin{tabular}{c c c c c c c}
    \toprule
    \multirow{2}{*}{Load} & \multicolumn{2}{c}{With BEO} & \multicolumn{2}{c}{Without BEO} & \\[0.2cm]
    \cmidrule{2-7}
     & $\bar{x}^{BEO}$ & $\sigma_{\bar{x}}^{BEO}$ & $\bar{x}^{noBEO}$ & $\sigma_{\bar{x}}^{noBEO}$ & Confidence \% & Quantity of Improvement \%\\
    \midrule
	76 & 2.73603 & 0.00853 & 2.83275 & 0.00683 & 100.00 & 3.41\\
    77 & 2.73761 & 0.00422 & 2.77537 & 0.00695 & 100.00 & 1.36\\
    78 & 2.72698 & 0.00250 & 2.74504 & 0.00213 & 99.99 & 0.66\\
    79 & 2.73320 & 0.00129 & 2.73955 & 0.00101 & 99.98 & 0.23\\
    80 & 2.73673 & 0.00110 & 2.74136 & 0.00066 & 100.00 & 0.17\\
    81 & 2.73244 & 0.00144 & 2.74462 & 0.00077 & 100.00 & 0.44\\
    82 & 2.73281 & 0.00169 & 2.74793 & 0.00126 & 100.00 & 0.55\\
    83 & 2.73722 & 0.00236 & 2.74594 & 0.00240 & 99.52 & 0.32\\
	\bottomrule
	\end{tabular}
    \caption{Data of improvement of power consumption from November 02, 2013 to November 07, 2013}
    \label{App:table02} 
  \end{center} 
\end{table}

\begin{table}[H]
\scriptsize
  \begin{center}
    \begin{tabular}{c c c c c c c}
    \toprule
    \multirow{2}{*}{Load} & \multicolumn{2}{c}{Without BEO} & \multicolumn{2}{c}{No BEO} & \\[0.2cm]
    \cmidrule{2-7}
     & $\bar{x}^{BEO}$ & $\sigma_{\bar{x}}^{BEO}$ & $\bar{x}^{noBEO}$ & $\sigma_{\bar{x}}^{noBEO}$ & Confidence \% & Quantity of Improvement \%\\
    \midrule
	72 & 2.71527 & 0.003884 & 2.768653 & 0.00263 & 100.00 & 1.93\\
	73 & 2.72383 & 0.002091 & 2.755741 & 0.00181 & 100.00 & 1.16\\
	74 & 2.72477 & 0.001643 & 2.749370 & 0.00116 & 100.00 & 0.89\\
	75 & 2.72900 & 0.001633 & 2.750293 & 0.00084 & 100.00 & 0.77\\
	76 & 2.73108 & 0.001807 & 2.753405 & 0.00077 & 100.00 & 0.81\\
	77 & 2.73184 & 0.002067 & 2.749418 & 0.00083 & 100.00 & 0.64\\
	78 & 2.73470 & 0.001581 & 2.753100 & 0.00084 & 100.00 & 0.67\\
	79 & 2.74096 & 0.001106 & 2.753691 & 0.00120 & 100.00 & 0.46\\
	80 & 2.74455 & 0.001184 & 2.747897 & 0.00177 & 94.18 & 0.12\\
	81 & 2.73922 & 0.001432 & 2.749539 & 0.00242 & 99.99 & 0.38\\
	82 & 2.73677 & 0.001302 & 2.751834 & 0.00210 & 100.00 & 0.55\\
	83 & 2.74342 & 0.001493 & 2.755530 & 0.00197 & 100.00 & 0.44\\
	84 & 2.74808 & 0.001741 & 2.767773 & 0.00368 & 100.00 & 0.71\\
	\bottomrule
	\end{tabular}
    \caption{Data of improvement of power consumption from November 08, 2013 to November 18, 2013}
    \label{App:table03} 
  \end{center} 
\end{table}

\begin{figure}[H]
  \centering
  \includegraphics[width=0.8\textwidth]{images/Appendix1/BEO_Ben_p1.pdf}
  \caption{Plot of improvement of power consumption from November 02, 2013 to November 07, 2013}
  \label{App:fig03}    
\end{figure}

\begin{figure}[H]
  \centering
  \includegraphics[width=0.8\textwidth]{images/Appendix1/BEO_Ben_p2.pdf}
  \caption{Plot of improvement of power consumption from November 08, 2013 to November 18, 2013}
  \label{App:fig04}    
\end{figure}

\paragraph{Improvement of power consumption by boiler load.}

Table \ref{App:table02} \& Figure \ref{App:fig03} show the improvement of power consumption with BEO and without BEO in the first period. Table \ref{App:table03} \& Figure \ref{App:fig04} show the improvement of power consumption with BEO and without BEO in the second period. The experimental results show that BEO improved the performance of power consumption with confidence larger 95\% at nearly all load. At 80 ton/h boiler load, BEO achieves the improvement of the performance of power consumption with confidence 94.18\%. The Figures \ref{App:fig03} \& \ref{App:fig04} show that BEO achieves good improvement in power consumption for boiler load below 78 ton/h and over 82 ton/h. Total improvement of power consumption in observed time (17 days) can be summarized as follows.

\begin{itemize}
\item The power consumption in total with BEO = 23167.32 MMBTU (one million British Thermal Unit)
\item The power consumption in total with BEO (equivalent calculation) = 23288.63 MMBTU
\end{itemize}

The experimental results show that the power consumption reduced in total. It means that boiler with the support of BEO improved the performance of power consumption approximately 0.52\%.

\paragraph{Estimated benefits by year of applying BEO.}

Total steam production of boiler at Phu My Fertilizer Plant is approximately 600,000 tons in 2013. The cost of the energy to produce a ton of steam is 2.75 MMBTU/T in average (according to the statistical data of the factory). The average of benefit for 0.52\% improved boiler efficiency can be explained as Table \ref{App:table01}.

\begin{table}[h!]
\scriptsize
  \begin{center}
    \begin{tabular}{p{0.5cm} p{1.5cm} p{1.5cm} p{1.5cm} p{1.5cm} p{1.8cm} p{1.2cm} p{1.2cm}}
    \toprule
    \multirow{2}{*}{Year} & Energy to produce one ton of steam without BEO (MMBTU/h) & Improvement $\%$ & Energy decreasing (MMBTU/h) &  Energy to produce one ton of steam with BEO (MMBTU/h) & Fuel cost (USD/MMBTU) & Benefits per ton (USD) & Total benefits per years (USD) \\[0.2cm]
    \midrule
    2013 & 2.75 & 0.52 & 0.0143 & 2.7357 & 6.56 & 0.09381 & 56,286.00\\
	2014 & 2.75 & 0.52 & 0.0143 & 2.7357 & 6.69 & 0.09567 & 57,402.00\\
	\bottomrule		
	\end{tabular}
    \caption{Estimated Benefit of BEO}
    \label{chap05:table01} 
  \end{center} 
\end{table}

In conclusion, in reality, the average natural gas price is about 6.56 USD/MMBTU in 2013 and about 6.69 USD/MMBTU in 2014, the total benefit estimated is about 57,000.00 USD per year. The benefits achieved prove the BEO soft sensor is effective for the boiler operation.


%\cleardoublepage
%\include{Appendix2}


