\chapter{BACKGROUND KNOWLEDGE}
\label{chap:background}
\paragraph{Chương 2} trình bày các kiến thức nền tảng cần thiết phục vụ cho quá trình thực hiện đề tài, bao gồm các nội dung:

\begin{itemize}
\item Giới thiệu về mạng học sâu, một phạm trù của Machine Learning

\item Định nghĩa, cấu trúc, cách hoạt động của mạng neural nhân tạo

\item Mở rộng của mạng neural nhân tạo - mạng neural tích chập - cấu trúc, các lớp, cơ chế tính toán và ứng dụng của nó trong việc xử lí dữ liệu có dạng hình ảnh

\item Các thuật toán nhận diện hình ảnh sử dụng mô hình neural tích chập, gồm: R-CNN, Fast R-CNN và Faster R-CNN
\end{itemize}

\section{MVC model}
\section{Database}
\section{Face recognition}
\section{Action recognition}
\section{Deep neural network}
\section{Cloud computing}
