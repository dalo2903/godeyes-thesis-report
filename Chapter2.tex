\chapter{BACKGROUND KNOWLEDGE}
\label{chap:background}
\paragraph{Chapter 2} Presentation of background knowledge necessary for the implementation process, including the content:

\begin{itemize}
\item Giới thiệu về mạng học sâu, một phạm trù của Machine Learning

\item Định nghĩa, cấu trúc, cách hoạt động của mạng neural nhân tạo

\item Mở rộng của mạng neural nhân tạo - mạng neural tích chập - cấu trúc, các lớp, cơ chế tính toán và ứng dụng của nó trong việc xử lí dữ liệu có dạng hình ảnh

\item Các thuật toán nhận diện hình ảnh sử dụng mô hình neural tích chập, gồm: R-CNN, Fast R-CNN và Faster R-CNN
\end{itemize}

\section{Crowd-sourcing}

\section{MVC model}
%% chưa parapharse https://developer.mozilla.org/en-US/docs/Web/Apps/Fundamentals/Modern_web_app_architecture/MVC_architecture %%
Model View Controller (MVC) is a software architecture pattern, commonly used to implement user interfaces: it is therefore a popular choice for architecting web apps. In general, it separates out the application logic into three separate parts, promoting modularity and ease of collaboration and reuse. It also makes applications more flexible and welcoming to iterations.

To make this a little more clear, let's imagine a simple shopping list app. All we want is a list of the name, quantity and price of each item we need to buy this week. Below we'll describe how we could implement some of this functionality using MVC.

\section{Database}
\section{Face recognition}
\section{Action recognition}
\section{Deep neural network}
\section{Cloud computing}
