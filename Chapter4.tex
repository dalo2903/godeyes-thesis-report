\chapter{SYSTEM DESIGN AND IMPLEMENTATION}

This chapter describes the implementation of the system which consists of: A social media website, an image processing module, a video processing module, a database. In \textit{System overview and requirement analysis} section, everything is briefly described as a whole. In other sections, structure and details of each component are defined more comprehensively.
 \section{System overview and requirement analysis}

\subsection{System overview}
Figure \ref{chap4:system_overview_basic} shows a concise overview of how the system operates. Users interact with the website through front-end. The input of users can come in the form of images, videos or label contribution. Back-end receives input and stores in the database. Concurrently, inputs are sent to image processing module/video processing module . Outputs from analysis server are sent back to back-end and then stored in database. The back-end is also responsible for obtaining appropriate contents from the database to display to users through front-end.

\begin{center}
    \begin{figure}[H]
    \centering
    \includegraphics[width=1\columnwidth]{images/chap4/system_overview_basic.png}
    \caption{An overview of the system}
    \label{chap4:system_overview_basic}
    \end{figure}
\end{center}

Figure \ref{chap4:system_architecture} describe the structure of the system architecture more specifically: Outlining the details of each component and interaction between them.

\begin{center}
    \begin{figure}[H]
    \centering
    \includegraphics[width=1\columnwidth]{images/chap4/system_architecture.png}
    \caption{System architecture}
    \label{chap4:system_architecture}
    \end{figure}
\end{center}

\subsection{Requirement analysis}
A requirements analysis is essential to any system. In this section, this thesis defines functional as well as non-function requirements for the system. Use cases and a system overview is also outlined in the following subsections.
\subsubsection{Types of users}
As with any sophisticated system, the proposed system has many types of user. Those type of users are:

\begin{itemize}
    \item \textbf{Guest users}: Users that have not verify their account by authentication methods.
    \item \textbf{Verified users}: Users that are verified with the system by authentication methods.
    \item \textbf{Admins}: Supervisors of the system. This title is given to people who are responsible for managing the system.
\end{itemize}
\subsubsection{Functional Requirements}
As stated in \ref{chap:solution}, this project would build a social media website for security. Such a website must have functions of a typical social media as well as additional functions for security. Those functions would include:
\begin{itemize}	
    \item Verified users and Guest users can view posts by other users. 
    \item Verified users could create posts with images or videos and includes location and time information in their posts. The website stores posts and image/videos in databases and analyze them through analyzing modules.
    \item Verified users could contribute to the system by labeling videos. Labels of videos are stored in the database and are used to train video classification modules.
    \item Verified users could subscribe to a location to get notifications about the suspicious videos from that location.
    \item When suspicious videos are detected, notifications are sent to Users subscribed to nearby locations.
    \item The system can recognize faces from images and logs their occurrences to the database.
\end{itemize}


\subsubsection{Non-functional Requirement}
In the scope of a thesis, the system has to acquire these requirements to perform processes that are stable and efficient for the user:
\begin{itemize}
	\item Reliability: The System is able to have 90\% uptime of Internet connection. It should have OAuth services via Google/Facebook(account kit) for authentication.
	\item Scalability: The website is able to handle many concurrent connections and manage a large amount of posts and store into NoSQL database .
	\item Usability: The website uses modern UI developed in HTML5. The user can use that UI easily to navigate between each page. It also has a fresh and well-designed interface for user to interact with
\end{itemize} 
\subsection{Use Cases}
\label{chap4:usecases}
\begin{center}
	\begin{figure}[H]
		\centering
		\includegraphics[width=0.75\columnwidth]{images/chap4/usecase.png}
		\caption{Use case diagram}
		\label{chap4:user_case_diagram}
	\end{figure}
\end{center}
% Please add the following required packages to your document preamble:
% \usepackage{multirow}
% Please add the following required packages to your document preamble:
% \usepackage{multirow}
% \usepackage[normalem]{ulem}
% \useunder{\uline}{\ul}{}
The use cases description clearly explain how these features work in the system, which are shown through the tables listed below:
\begin{table}[H]
\begin{tabular}{|P{5cm}|L{10cm}|}
		\hline
	Name							&   UC01 -   Log in website   \\ \hline
	Description 	 				&   Users will be prompted to login with their account information before they can use the system.   \\ \hline
	Actor 							&  User and Admin       \\ \hline
\multirow{3}{*}{Preconditions} 		&\tabitem The user has an website account \\ \
									&\tabitem The user is trying to log in with their account  \\ \
									&\tabitem The user is not already logged in with website  \\ \hline
\multirow{2}{*}{Postconditions}	 	&\tabitem The user is logged in to the system     \\ 
									&\tabitem The user has access to the functions of the
									system\\ \hline
\multirow{9}{*}{Path} 				&\tabitem Primary path:    \\
									& 1.User accesses the page \\ 
									& 1.1 The system prompts the user for their 
									account sign up. \\
									& 1.2 The user enters their username and
									password. \\
									& 1.3 The system authenticates the  login
									with website \\
									& 1.4 The user gains access to the systems
									functionality \\ \cline{2-2} 
									&\tabitem Alternate path  \\
									& 2. Invalid account user or banned \\
									& 2.1 User already logged in with website \\ \hline
									
	\end{tabular}
\caption{Log in website }
\end{table}
\begin{table}[H]
	\begin{tabular}{|P{5cm}|L{10cm}|}
		\hline
		Name						&   UC02 - View post in website     \\ \hline
		Description 	 			&   Users can view detail in posts   \\ \hline
		Actor 						&  	User and Admin       \\ \hline
		Preconditions 				& 	System check the role of the user to provide the following function 						 \\ \hline
		Postconditions	 			&							 \\ \hline 
\multirow{9}{*}{Path} 						&\tabitem Primary path:    \\
									& 1.Unverified-user press to title of the post \\ 
									& 1.1 The system inquires the user for their roles
									account . \\
									& 1.2 The user can see information of post and commend \\
									& 2. Verified-user and Moderator press to title of the post \\
									& 2.1 The system checks the user for their roles
									account . \\
									& 2.2 Verified-user and Moderator can use feature as commend and report post \\ \cline{2-2} 
									&\tabitem Alternate path  \\
									& 3. View post by tab My Posts in Profile user\\ \hline
	\end{tabular}
\caption{View post in website}
\end{table}
\begin{table}[H]
	\begin{tabular}{|P{5cm}|L{10cm}|}
		\hline
		Name						&   UC03 - Create new post in website       \\ \hline
		Description 	 			&   Users can create new data for everyone to view on website's new feed \\ \hline
		Actor 						&  	Verified-user and Admin       \\ \hline
		Preconditions 				& 	The system checks user’s phone number whether is verified  						 \\ \hline
\multirow{3}{*}{Postconditions} 	&	\tabitem Server stores data to database \\
									&	\tabitem Create new location on website for user to subscribe\\
									&   \tabitem Send a request api to video classification service to analyze data and get a result to notify user					 \\ \hline 
									
\multirow{4}{*}{Path} 				&\tabitem Primary path:    \\
									& 1.Verified-user and Admin access post creation page    \\ 
									& 1.1 The system inquires the user for their roles 
									account and user’s phone number. \\
									& 1.2 The user fills in post's title,  pictures/video, location when user capture that data  \\  \hline
	\end{tabular}
\caption{Create new post}
\end{table}
\begin{table}[H]
	\begin{tabular}{|P{5cm}|L{10cm}|}
		\hline
		Name						&   UC04 -Contribute label of videos         \\ \hline
		Description 	 			&   Users can update label of all videos exist on website \\ \hline
		Actor 						&  	Verified-user and Admin       \\ \hline
		Preconditions 				& 	The system checks user’s phone number whether is verified  	 \\ \hline	
\multirow{2}{*}{Postconditions} 	&	\tabitem Server updates the user's vote \\
									&   \tabitem Server stores id of the user \\ \hline 									
\multirow{4}{*}{Path} 				&	\tabitem Primary path:    \\
									& 1.Verified-user and Admin access post labeling page    \\ 
									& 1.1 The system inquires the user for their roles 
									account and user’s phone number. \\
									& 1.2 The user clicks available label button \\  \hline
	\end{tabular}
\caption{Contribute label of videos}
\end{table}
\begin{table}[H]
	\begin{tabular}{|P{5cm}|L{10cm}|}
		\hline
		Name						&   UC05 -Delete post on website         \\ \hline
		Description 	 			&   Users can delete their videos  \\ \hline
		Actor 						&  	Verified-user and Admin       \\ \hline
		Preconditions 				& 	The system checks role of the user  	 \\ \hline	
\multirow{2}{*}{Postconditions} 	&	\tabitem Server deletes information of post in database \\
									&   \tabitem Server deletes image/video in cloud storage \\ \hline 									
\multirow{9}{*}{Path} 				&	\tabitem Primary path:    \\
									& 1.Verified-user access profile of user page    \\ 
									& 1.1 The system inquires the user for their roles 
		account. \\
									& 1.2 The user clicks on "MY POSTS" tab \\ 
									& 1.3 The user clicks on "DELETE" button \\
									& 2.The moderator access admin's page\\
									& 2.1. The system check authorization of admin \\
									& 2.2  The admin clicks on "REPORTED POSTS" tab \\
									& 2.3. The admin clicks "Delete Post" button\\ \hline
	\end{tabular}
\caption{Delete post }
\end{table}
\begin{table}[H]
	\begin{tabular}{|P{5cm}|L{10cm}|}
		\hline
		Name						&   UC06 -Subscribe location       \\ \hline
		Description 	 			&   Users can subscribe location to get notifications   \\ \hline
		Actor 						&  	Verified-user and Admin       \\ \hline
		Preconditions 				& 	The system checks role of the user  	 \\ \hline	
\multirow{3}{*}{Postconditions} 	&	\tabitem Server stores information of that user to location's collection in database \\
									&   \tabitem Server saves information of the location to user's collection in database \\ 
									& 	\tabitem The user receives notification after server analyze data	\\ \hline								
\multirow{5}{*}{Path} 				&	\tabitem Primary path:    \\
									& 1.Verified-user accesses profile of user  page   \\ 
									& 1.1 The system inquires the user for their roles 
									account. \\
									& 1.2 The user clicks on "All LOCATION" tab \\ 
									& 1.3 The user clicks on "SUBSCRIBE" button \\ \hline

	\end{tabular}
\caption{Subscribe location}
\end{table}
\begin{table}[H]
	\begin{tabular}{|P{5cm}|L{10cm}|}
		\hline
		Name						&   UC07 -Unsubscribe location       \\ \hline
		Description 	 			&   Users can unsubscribe location to stop getting notifications \\ \hline
		Actor 						&  	Verified-user and Admin       \\ \hline
		Preconditions 				& 	The system checks role of the user  	 \\ \hline	
\multirow{3}{*}{Postconditions} 	&	\tabitem Server deletes information of that user in location's collection in database \\
									&   \tabitem Server deletes information of the location in user's collection in database \\ 
									& 	\tabitem the user stop receiving notifications from that location	\\ \hline								
\multirow{7}{*}{Path} 				&	\tabitem Primary path:    \\
									& 1.Verified-user accesses profile of user page  \\ 
									& 1.1. The system inquires the user for their roles 
									account. \\
									& 1.2. The user clicks on "SUBSCRIBED LOCATION" tab \\ 
									& 1.3. The user clicks on "UNSUBSCRIBE" button \\ \cline{2-2}
									&   \tabitem Alternate path: \\
									& 1.4. The user clicks on "All LOCATION" tab \\
									& 1.5. The user clicks on "UNSUBSCRIBE" tab\\ \hline

	\end{tabular}
	\caption{Unsubscribe location}
\end{table}
\begin{table}[H]
	\begin{tabular}{|P{5cm}|L{10cm}|}
		\hline
		Name						&   UC08 -View notification       \\ \hline
		Description 	 			&   Users can view notifications of subscribed locations \\ \hline
		Actor 						&  	Verified-user and Admin       \\ \hline
		Preconditions 				& 	The system checks role of the user  	 \\ \hline	
		Postconditions 				&	\\ \hline								
		\multirow{10}{*}{Path} 		&	\tabitem Primary path:    \\
									& 1.Verified-user clicks on bell icon on header     \\ 
									& 1.1. The system inquires the user for their roles 
									account. \\
									& 1.2. Header shows notification box dropdown\\
									& 1.3. The user clicks on notification \\
									& 1.4. The website navigates user's view to notification's detail\\
									& 2.Verified-user accesses profile of user page \\
									& 2.1. User clicks on "NOTIFICATIONS" tab\\
									& 2.2. The user clicks on notification \\
									& 2.3. The website navigates user's view to notification's detail\\ \hline
	\end{tabular}
	\caption{View notification}
\end{table}
\begin{table}[H]
	\begin{tabular}{|P{5cm}|L{10cm}|}
		\hline
		Name						&   UC09 -Report post       \\ \hline
		Description 	 			&   Users can report posts with incorrect information \\ \hline
		Actor 						&  	Verified-user and Admin       \\ \hline
		\multirow{2}{*}{Preconditions} 				& 	\tabitem The system checks role of the user  	 \\ 
									&   \tabitem The system checks whether the user yet to report the post  \\
									\hline	
		\multirow{2}{*}{Postconditions} &\tabitem	The system saves id of user to post's collection    \\   								
									&  \tabitem The system saves comment about that report to post's collection\\ \hline								
		\multirow{5}{*}{Path} 		&	\tabitem Primary path:    \\
									& 1.Verified-user clicks on report button of each post in news feed     \\ 
									& 1.1. Comment pop up appear for user to notify admin. \\
									& 2.Verified-user access post detail page \\
									& 2.1 The user clicks on report button \\ \hline
									
	\end{tabular}
	\caption{Report post}
\end{table}

\begin{table}[H]
\begin{tabular}{|P{5cm}|L{10cm}|}
	\hline
	Name						&   UC10 -View record       \\ \hline
	Description 	 			&  Admin can view record of all recognized person   \\ \hline
	Actor 						&  Admin       \\ \hline
	Preconditions 				& The system checks role of the user  	 \\ \hline	
	Postconditions 				&    \\ \hline								
	\multirow{2}{*}{Path} 		&	\tabitem Primary path:    \\
								& 1.  Admin accesses admin's page\\
								& 1.1 Admin clicks on "RECORDS" tab    \\ \hline
	
\end{tabular}
\caption{View records}
\end{table}
% \subsubsection{Main function of the system: Notify users about suspicious activity using data from posts.}

% \begin{center}
% 	\begin{figure}[H]
% 		\centering
% 		\includegraphics[width=0.7\columnwidth]{images/chap4/createpostflowchart.png}
% 		\caption{Create post flowchart.}
% 	\end{figure}
% \end{center}

% Specifically, users create post with image/video to provide data to back-end. Post also contains other information such as author, location, time. Back-end save all information from post and then deliver data to relevant module to analyze. When suspicious activity detected after analyzing, back-end notify users who subscribe to the exact location and users who subscribe to locations that are in a diameter of 1 kilometer from that location.

\section{Technologies}
\subsection{Node.js and Node Package Manager (npm)}
Node.js (Node) is a JavaScript runtime built on Chrome's V8 JavaScript engine (cite). Node supports executing JavaScript on server-side. Ryan Dahl was created Node.js in 2009. Node.js Foundation is in charge of the development of Node.js. After nine years since its release, the latest LTS version of Node.js is 10.14.2 (includes npm 6.4.1). “Node.js operates on a single thread, using non-blocking I/O calls, allowing it to support tens of thousands (cite) of concurrent connections held in the event loop.” Although JavaScript is single-threaded, thanks to the Event loop, Node.js can implement asynchronous I/O operations. The Event loop will transfer operations to the system kernel. “Since most modern kernels are multi-threaded, they can handle multiple operations executing in the background. When one of these operations completes, the kernel tells Node.js so that the appropriate callback may be added to the poll queue to executed eventually”. By utilizes non-blocking I/O, Node.js skips the waiting time for I/O calls, which is much higher than processing time.
\subsection{Keras}
Keras is a high-level framework for building Deep learning model build on top of Tensorflow or Theano library.\\
Keras provides APIs that are more friendly for new learner compared to Tensorflow. Besides that, Keras also has useful functions for preprocessing data.
\subsection{Gunicorn}
Gunicorn is a Python Web Server Gateway Interface (WSGI) HTTP server. A WSGI recieve requests from web servers and foward them to Python applications.
Gunicorn helps separating the operation between web servers and applications, making applications more portable. Moreover, Gunicorn provides features for deployment such as maximum of requests for each worker, number of workers, making it useful for this thesis.

\section{Database}
The project database divides into two different parts: \textbf{Cloud storage} for visual data such as image/video and a \textbf{NoSQL database} for other information. The primary target is to reduce website loading time. Let take an example, if files are stored directly on the back-end server. When many users request a file simultaneously, the server with limited bandwidth will cause delay. “53\% of mobile site visitors leave a page that takes longer than three seconds to load” – \href{https://think.storage.googleapis.com/docs/mobile-page-speed-new-industry-benchmarks.pdf}{Google}. If these files stored on cloud storage, the client will be served by the cloud storage provider, which has higher availability.
\subsection{Cloud storage}
Google is stepping up as a reliable cloud storage provider. Google Cloud Storage is a service within the Google Cloud Platform. It provides unified object storage. 
Google Cloud Storage is chosen for its advantages:
\begin{itemize}
\item Good documentation: hundreds of pages in total, including a detailed API Reference guide.
\item Good prices: around 0,020 USD per GB/month with the Regional class and 0,007 USD per GB/month with the Coldline class, cheaper compared to other cloud storage provider.
\item Different storage classes for each necessity: Regional (frequent use), Nearline (infrequent use) and Coldline (long-term storage).
\item Many regions available to store data: North America, South America, Europe, Asia and Australia.
\item One of the best free layers in the industry. \$300 free credit to get started with any Google Cloud Platform product during the first year. Afterwards, 5 GB of Storage free to use forever.
\end{itemize}
Objects needed to be stored are image and video. Before being stored, data is renamed to a unique ID for later retrieval. Additionally, image is resized to reduce capacity for better performance.
\subsection{NoSQL database}
NoSQL databases were created in response to the limitations of traditional relational database technology. When compared against relational databases, NoSQL databases are more scalable and provide superior performance, and their data model addresses several shortcomings of the the relational model.
The advantages of NoSQL include being able to handle:
\begin{itemize}
\item Large volumes of structured, semi-structured, and unstructured data
\item Agile sprints, quick iteration, and frequent code pushes.
\item Object-oriented programming that is easy to use and flexible.
\item Efficient, scale-out architecture instead of expensive, monolithic architecture.
\end{itemize}
Hence, \textbf{MongoDB} is used to implement the database in this project.
\subsubsection{Database overview}
Figure \ref{chap4:database_overview} displays the structure of the database which includes collections and relationships between them.
There are a total of 8 collections in the database: User, Visual data, Location, Notification, Person, Post, Record, Role. 
\begin{center}
	\begin{figure}[H]
		\centering
		\includegraphics[width=1\columnwidth]{images/chap4/Model.png}
		\caption{An overview of the database}
		\label{chap4:database_overview}
	\end{figure}
\end{center}
\cleardoublepage
\subsubsection{Collections}
This section contains descriptions of collections along with their schema.\\

\textbf{User} collection contains basic information of users.
\begin{center}
	\begin{figure}[H]
		\centering
		\includegraphics[width=0.5\columnwidth]{images/chap4/User.png}
		\caption{User collection}
	\end{figure}
\end{center}
\cleardoublepage

\textbf{Visual data} contains information of files. This collection is synchronized with the file storage upon upload, update or deletion of files.

\begin{center}
	\begin{figure}[H]
		\centering
		\includegraphics[width=1\columnwidth]{images/chap4/Visual.png}
		\caption{Visual Data collection}
	\end{figure}
\end{center}
\cleardoublepage

\textbf{Location} contains information of locations. This collection utilizes Geospatial Query feature of MongoDB. Geospatial store each location as a point on the map with their longitude and latitude. Spatial queries such as finding locations near a point or finding locations within a certain radius of a point can be made.
\begin{center}
	\begin{figure}[H]
		\centering
		\includegraphics[width=1\columnwidth]{images/chap4/Location.png}
		\caption{Location collection}
	\end{figure}
\end{center}
\cleardoublepage

\textbf{Notification} contains notifications to users. 
\begin{center}
	\begin{figure}[H]
		\centering
		\includegraphics[width=0.6\columnwidth]{images/chap4/Notification.png}
		\caption{Notification collection}
	\end{figure}
\end{center}
\cleardoublepage

\textbf{Person} store identification of images data. Each time a new person is identified, that person is stored into the collection.
\begin{center}
	\begin{figure}[H]
		\centering
		\includegraphics[width=0.7\columnwidth]{images/chap4/Person.png}
		\caption{Person collection}
	\end{figure}
\end{center}
\cleardoublepage

\textbf{Post} stores posts of users.
\begin{center}
	\begin{figure}[H]
		\centering
		\includegraphics[width=0.7\columnwidth]{images/chap4/Post.png}
		\caption{Post collection}
	\end{figure}
\end{center}
\cleardoublepage

\textbf{Record} collection logs every occurrences of a person identified by the system with timestamps and locations.
\begin{center}
	\begin{figure}[H]
		\centering
		\includegraphics[width=0.7\columnwidth]{images/chap4/Record.png}
		\caption{Record collection}
	\end{figure}
\end{center}
\cleardoublepage

\textbf{Role} collection stores roles of users in the system.
\begin{center}
	\begin{figure}[H]
		\centering
		\includegraphics[width=0.7\columnwidth]{images/chap4/Role.png}
		\caption{Role collection}
	\end{figure}
\end{center}
\cleardoublepage
\subsubsection{Relationships}
The following table contains the relationship between collections.
\begin{table}[H]
	\begin{tabular}{|l|l|}
		\hline
		\textbf{Children field}                             & \textbf{Parent field} \\ \hline
		Location.subscribers.{[}0{]}                        & User.\_id             \\ \hline
		Notification.to                                     & User.\_id             \\ \hline
		Notification.records.{[}0{]}                        & Record.\_id           \\ \hline
		Notification.data                                   & Visual Data.\_id      \\ \hline
		Notification.location                               & Location.\_id         \\ \hline
		Person.userCreated                                  & User.\_id             \\ \hline
		Person.datas.{[}0{]}                                & Visual Data.\_id      \\ \hline
		Person.location                                     & Location.\_id         \\ \hline
		Post.userCreated                                    & User.\_id             \\ \hline
		Post.datas.{[}0{]}                                  & Visual Data.\_id      \\ \hline
		Post.location                                       & Location.\_id         \\ \hline
		Post.reported.{[}0{]}                               & User.\_id             \\ \hline
		Record.data                                         & Visual data.\_id      \\ \hline
		Record.location                                     & Location.\_id         \\ \hline
		Record.personId                                     & Person.\_id           \\ \hline
		User.personId                                       & Person.\_id           \\ \hline
		User.address                                        & Location.\_id         \\ \hline
		User.subscribed.{[}0{]}                             & Location.\_id         \\ \hline
		Visual Data.labels.{[}0{]}.user                     & User.\_id             \\ \hline
		Visual Data.identifyResult.persons.{[}0{]}.personId & Person.\_id           \\ \hline
		Visual Data.location                                & Location.\_id         \\ \hline
		User.role                                           & Role.type             \\ \hline
	\end{tabular}
	\caption{Relationships of collections.}
\end{table}
\cleardoublepage
Example:
\begin{table}[H]
	\begin{tabular}{|l|l|}
		\hline
		Children field               & Parent field \\ \hline
		Location.subscribers.{[}0{]} & User.\_id    \\ \hline
	\end{tabular}
\end{table}
The element of the field \textit{subscribers}(array) of collection \textbf{Location} references field \textit{\_id} of collection \textbf{User}.
\section{Social media website}
To implement the website, Model-View-Controller (MVC) pattern is used for its advantages:
\begin{itemize}
\item Faster development process.
\item Ability to provide multiple views.
\item Support for asynchronous technique.
\item Modification does not affect the entire model.
\item MVC model returns the data without formatting.
\item SEO friendly Development platform.
\end{itemize}
In the MVC development, controller receives all requests for the application and then control the model to prepare any necessary information for the view. The view uses that data prepared by the controller to bring the final output.
\subsection{Views}
\subsubsection{Overview}
Figure \ref{chap4:sitemap} display the overview of pages of the front-end and corresponding function can be found in each page.
\begin{center}
    \begin{figure}[H]
    \centering
    \includegraphics[width=17cm,height=5cm]{images/chap4/sitemap-edit2.png}
    \caption{Sitemap}
    \label{chap4:sitemap}
    \end{figure}
\end{center}
\subsubsection{Examples}
	\textbf{Header}
\\
Header is a navigation bar which is displayed in every views. Navigation bar only contains "About us", "Sign in", "Sign up" buttons by default.
\begin{figure}[!htb]
\minipage{0.325\textwidth}
  \includegraphics[width=\linewidth]{images/chap4/header_not_login_mb.jpg}
\endminipage\hfill
\minipage{0.05\textwidth}
\endminipage\hfill
\minipage{0.625\textwidth}
  \includegraphics[width=\linewidth]{images/chap4/header_not_login.png}
\endminipage
\end{figure}
\cleardoublepage
Logging in add additional buttons such as "Create post", "Label videos",  "Profile", "Log out" and a notification dropdown represented by a bell icon to the bar. In case of having unseen notifications, the bell icon will have an orange-red color.
\\
\begin{figure}[!htb]
\minipage{0.325\textwidth}
  \includegraphics[width=\linewidth]{images/chap4/header_mb.jpg}
\endminipage\hfill
\minipage{0.05\textwidth}
\endminipage\hfill
\minipage{0.625\textwidth}
  \includegraphics[width=\linewidth]{images/chap4/header.png}
\endminipage
\end{figure}
\cleardoublepage
\textbf{Homepage}
\\
News feed which are posts from everyone is visible in the middle of the homepage. A post is displayed with title, author, time, location and visual data. Additionally, there is a sidebar with popular locations. Clicking a location from the sidebar leads to another view that show only posts from that location. Blank space is reserved for advertising to earn income in the future.
\begin{figure}[!htb]
\minipage{0.325\textwidth}
  \includegraphics[width=\linewidth]{images/chap4/homepage_mb.jpg}
\endminipage\hfill
\minipage{0.05\textwidth}
\endminipage\hfill
\minipage{0.625\textwidth}
  \includegraphics[width=\linewidth]{images/chap4/homepage.png}
\endminipage
\end{figure}
\cleardoublepage
\textbf{Sign in}
\\
Sign in form also has buttons for users to login with Google and Facebook account.
\begin{figure}[!htb]
\minipage{0.325\textwidth}
  \includegraphics[width=\linewidth]{images/chap4/signin_mb.jpg}
\endminipage\hfill
\minipage{0.05\textwidth}
\endminipage\hfill
\minipage{0.625\textwidth}
  \includegraphics[width=\linewidth]{images/chap4/signin.png}
\endminipage
\end{figure}
\cleardoublepage
\textbf{Sign up}
\\
Sign up form has general rules when using the website. Rule violation will result in banning violator's account.
\begin{figure}[!htb]
\minipage{0.325\textwidth}
  \includegraphics[width=\linewidth]{images/chap4/signup_mb.jpg}
\endminipage\hfill
\minipage{0.05\textwidth}
\endminipage\hfill
\minipage{0.625\textwidth}
  \includegraphics[width=\linewidth]{images/chap4/signup_form.png}
\endminipage
\end{figure}
\cleardoublepage
\textbf{Label video}
\\
Label video view contains a list of videos and two buttons "Suspicious" and "Not suspicious" for users to label. A user can only label a video once. Labeled videos are used for training video classifier module.
\begin{figure}[!htb]
\minipage{0.325\textwidth}
  \includegraphics[width=\linewidth]{images/chap4/label_mb.jpg}
\endminipage\hfill
\minipage{0.05\textwidth}
\endminipage\hfill
\minipage{0.625\textwidth}
  \includegraphics[width=\linewidth]{images/chap4/label_video.png}
\endminipage
\end{figure}
\cleardoublepage
\textbf{Create post}
\\
Create post form includes title, a file selector to select image/video, an embedded Google map to select location. After creating post, location of created post is available for other users to subscribe.
\begin{figure}[!htb]
\minipage{0.325\textwidth}
  \includegraphics[width=\linewidth]{images/chap4/create_post_mb.jpg}
\endminipage\hfill
\minipage{0.05\textwidth}
\endminipage\hfill
\minipage{0.625\textwidth}
  \includegraphics[width=\linewidth]{images/chap4/create_post.png}
\endminipage
\end{figure}
\cleardoublepage
\textbf{Profile}
\\
Profile view contains user's basic info and mobile phone field for verification. Besides, it also has 4 tabs:  
\begin{itemize}
\item My posts
\item Notifications
\item Subscribed locations: To unsubscribe a location.
\item All locations: To subscribe to a new location.
\end{itemize}
\begin{figure}[!htb]
\minipage{0.325\textwidth}
  \includegraphics[width=\linewidth]{images/chap4/profile_mb.jpg}
\endminipage\hfill
\minipage{0.05\textwidth}
\endminipage\hfill
\minipage{0.625\textwidth}
  \includegraphics[width=\linewidth]{images/chap4/profile.png}
\endminipage
\end{figure}
\cleardoublepage
\textbf{Administration}
\\
Administration consists of 5 tabs:
\begin{itemize}
\item Locations
\item Visual data
\item Record: To look up an identified person's location history.
\item All users: To ban or unban a user.
\item Reported posts: To review reports and decide whether to delete post.
\end{itemize}
There is also an "Add person" button to add new person with name for system to identify in the future. An unidentified person is shown as "unknown".
\begin{figure}[!htb]
\minipage{0.325\textwidth}
  \includegraphics[width=\linewidth]{images/chap4/admin_mb.jpg}
\endminipage\hfill
\minipage{0.05\textwidth}
\endminipage\hfill
\minipage{0.625\textwidth}
  \includegraphics[width=\linewidth]{images/chap4/admin.png}
\endminipage
\end{figure}
\subsection{Controllers}
In MVC pattern, the Controller receives user input, manipulates Model, and then responds to the user. Following is the list of implemented controllers in this thesis.
\begin{itemize}
	\item Authentication Controller: manages the flow of user sign in and sign up. A user can choose either to login with traditional account or social account. With the classical way, the user has to provide his/her email and password. If the user prefers social login, the website currently supports sign in via Facebook and Google+.
	\item Dataset Collector: downloads dataset (a collection of labeled videos) as the input for the training process.
	\item Email Controller: One way to interact with users is sending emails. This controller sends email about a notification via SendGrid, an email delivery service.
	\item Face Controller: The core of Face Recognition Module is Microsoft Azure Face API, a cognitive service that provides algorithms for recognizing human faces in images (cite). Face Controller contains a collection of functions to request Microsoft Azure Face API.
	\item Identify Controller: This controller utilize Face Controller to combine basic features of Face API to do more complicated jobs. For example, to identify faces from an image, \textit{Detect} API first has to be called to generate faceids. Those faceids can then be used to identify images.
	\item Location Controller: This controller has two purposes: manipulate location and location subscriber. It controls in which condition a location will be created. Additionally, Each location has a list of subscriber who  will be notified about suspicious activity.
	\item Notification Controller:  take care of creating notification and send to correct users; Checking status of notification whether it has been seen by the user.
	\item Person Controller: This controller takes responsibility for managing Person and synchronizing with Person of Microsoft Azure Face API.
	\item Post Controller: is one of the essential controllers in this system. This controller together with Upload Controller, manage the flow of post creating.
	\item Record Controller: This controller store every occurrences of a person along the timestamps and location into the database.
	\item Upload Controller: When a user creates a post, this controller takes a picture, or video then uploads to Google Cloud Storage and finally returns public URL of that resource.
	\item User Controller: is in charge of controlling users: ban, unban user; subscribe, unsubscribe a location; phone number verification.
	\item Visual Data Controller: For each image or video uploaded to the Google Cloud Storage, this controller will create a comparable visual data with related data (location, public URL, …). Verified users can be able to decide whether a video is suspicious or not. This controller also take the identified result of visual data after analyzed by Microsoft Azure Face API.    
\end{itemize}

This section describe some sequences of use cases defined in section \ref{chap4:usecases}.
\begin{itemize}
	\item \textbf{Login} (UC01)
	\begin{center}
		\begin{figure}[H]
		\centering
		\includegraphics[width=0.75\columnwidth]{images/chap4/login_sequence.png}
		\caption{Sequence diagram for Login operation}
		\end{figure}
	\end{center}
	\item \textbf{View Post} (UC02)
	\begin{center}
		\begin{figure}[H]
		\centering
		\includegraphics[width=0.75\columnwidth]{images/chap4/post_sequence.png}
		\caption{Sequence diagram for viewing Posts}
		\end{figure}
	\end{center}
	\item \textbf{Create Post} (UC03)
	\begin{center}
		\begin{figure}[H]
		\centering
		\includegraphics[width=1\columnwidth]{images/chap4/createpost_sequence.png}
		\caption{Sequence diagram for Post creation with images}
		\end{figure}
	\end{center}
	\begin{center}
		\begin{figure}[H]
		\centering
		\includegraphics[width=1\columnwidth]{images/chap4/createpost_sequence_video.png}
		\caption{Sequence diagram for Post creation with videos}
		\end{figure}
	\end{center}

\end{itemize}

\subsection{Main functions}
\subsubsection{Receive and save data}
After a user is verified, he/she a has a right to create a new post. First, he/she will provide the required information for a new post: title, location, and videos or images. Next, post's create form will be sent to the Social Media Website.  After the Social Media Website receiving the request, it takes out visual data, uploads to a pre-defined bucket of Google Cloud Storage, and returns its public URL. Then, it will insert new visual data in NoSQL Database on mLab with that URL. Finally, a new post will be created and be labeled by other users or be analyzed later.
\subsubsection{Label video}
This function of the website collect labels of videos from users, then derives the final label of a videos based on what the users voted.To use this function, contributors go to a page that contains a list of video and their available labels. For each video, contributors can what it and choose the appropriate label based on the content.

\textbf{Storing label of users}\\
Labels of each video are store inside the Visual Data collection as an array of subdocuments. Each subdocument contains users' label and their id. Storing users' id prevent them from voting multiple times.

\textbf{Method of getting labels of videos}\\
Labels of videos are computed at the time of making the dataset using \textbf{Dataset Collector Controller}. For the sake of simplicity, the final labels of videos is taken by the majority. For each video, to compute its label, the system count the total amount of user voting on each label, then take the label that has the highest amount of voters. For example, if there are 4 users vote on \textit{label A} and 6 users vote on \textit{label B} for a video, then the final label of that video would be \textit{label A}. 

\subsubsection{Notify user about anomaly}
\begin{center}
    \begin{figure}[H]
    \centering
    \includegraphics[width=0.7\columnwidth]{images/chap4/createpostflowchart.png}
    \end{figure}
\end{center}
Only posts have video data are used to notify users. Suppose there is a user A who subscribe location called "1". When any user creates post with location which is exactly "1" or in a diameter of 1 kilometer from "1" and video data from that post is classified as "Suspicious", the system is going to notify user A.
\section{Face recognition module}
The facial recognition module does the job of identifying faces inside images, uses Microsoft Azure Face API as its core.
\begin{center}
    \begin{figure}[H]
    \centering
    \includegraphics[width=1\columnwidth]{images/chap4/face-api-homepage.PNG}
    \caption{Microsoft Azure Face API hompage}
    \end{figure}
\end{center}
\subsection{Face API}
There is two way to use Azure Face Service: client Face SDK and Azure Face REST API. This thesis applies the second one. Both ways require a Face API subscription key in order to communicate with Microsoft Cognitive Server. This thesis applies the second one. The Face endpoint URL will be in the following format:
\begin{center}
\textit{https://[location].api.cognitive.microsoft.com/face/v1.0}
\end{center}
The location part will be replaced by the region of the subscription key (visit \href{https://westus.dev.cognitive.microsoft.com/docs/services/563879b61984550e40cbbe8d/operations/563879b61984550f30395236}{Face API docs} for a full list of all available regions). The Face API version currently is \textit{v1.0}.
Let take a look of some basic concepts for Face API from its \textit{Glossary} (cite):
\begin{itemize}
\item Face: is a result of a detected face, contains a unified identity (Face ID), face location in an image (Face Rectangle), and optional face attributes. Face ID will expire in 24 hours after detection call.
\item Persisted Face: similar to Face, but is persisted and will not expire.
\item Person: is an object which includes its identity (Person ID), a bunch of Persisted Face of that person, and other related data.
\item Person Group: is a set of Persons and is the unit of Identification. A Person Group must be specified in order to identify a face. Each face will be recognized independently. Before the Identification, the Person Group should be trained successfully.
\begin{center}
    \begin{figure}[H]
    \centering
    \includegraphics[width=1\columnwidth]{images/chap4/face-api-person-group.jpg}
    \caption{An example of Person Group}
    \end{figure}
\end{center}
\item Person Group - Create: This API will create a new Person Group with a unique Person Group ID, name, and user-provided user data (optional). This Person Group will contain uploaded person data, including face images and face recognition features.
\item Person Group - Train: In order to perform a face identification from a Person Group, that Person Group should be trained by calling this API. This API will pre-process the Person Group to ensure Identification performance.
\item Person Group Person – Create: This API will create a Person in a Person Group. It requires Person Name and Person Group ID. If a Person created successfully, it would return a new Person ID.
\item Person Group Person – Add Face: This API adds a Face image for a Person in a Person Group. It will return a Persisted Face ID after added. This Persisted Face will be used later for Identification or Verification request.
\item Face - Detect: This API accepts image (in binary data or image’s URL) as input, detects human faces in that picture; returns face coordinates and attributes. It can detect up to 64 faces in a photo, and best for frontal and near-frontal faces. Each detected face will be assigned a unique Face ID and will be used in Face - Identify.
\item Face - Identify: This API takes an array of Face ID (query faces), compute similarities with all Faces in that Person Group. After that, it returns identified results, which is a set of Persons with confidence.
\end{itemize}
\subsection{Operation}
As mentioned above, HTTP requests are sent to Face REST API to be able to use the service. The main usage of Face API would be to Identify faces within images. Figure \ref{chap4:face_identify} describe the sequence for identifying faces.
\begin{center}
    \begin{figure}[H]
    \centering
    \includegraphics[width=0.75\columnwidth]{images/chap4/face_identify.png}
	\caption{Sequence diagram for face identification. }
	\label{chap4:face_identify}
    \end{figure}
\end{center}

\section{Video classifier module}
Video classifier module takes URL of videos and return their classification. This section describe the deployment of this module as well as the running environment.
\subsection{Deploying the video classifier system}
After training the model and saving the weights, to use the model for prediction, this thesis proposes building an HTTP server and provide an API for other components to send requests and receive the result. \\
Flask framework is used to implement this HTTP server because of its fast performance and is written in Python which helps communication with Keras APIs easier. \\
The HTTP server does the following jobs or every coming video: Extracting frames from the video, detecting frames that has people faces in it, classify that video, send classification results along with frames containing faces to the web server.
\begin{center}
    \begin{figure}[H]
    \centering
    \includegraphics[width=1\columnwidth]{images/chap4/server_sequence.png}
    \caption{Sequence diagram of the HTTP server}
    \end{figure}
\end{center}
\subsubsection{Extracting frames}
For every video, approximately 14 frames is extracted using OpenCV. First, the total number of frame and taken from the video. Then, interval of frames is calculated using the following formula:
\begin{center}
$interval = \frac{total\_frames}{frames\_to\_extract}$
\end{center}
Extracted frames are frames that are the multiple of $interval$. For example, when there are a total of
42 frames and there are 14 frames to extract, extracted frames are the multiple of 3, i.e.,the 3rd, 6th, 9th, .. frame. Each frames would then be passed through a pretrained VGG16 network to extract features. Using this method, frames would be extracted uniformly across the video.
\subsubsection{Detecting faces in videos}
The server also has to send frames of videos back to the web server for further identification. To save resources, only frames that contain faces within would be sent. For each frame, a Haar Cascade classifier is used to detect if there are faces within that frame, then compile them into a list of frames. Those frames would be send back to the web server along with the classification result.
\subsubsection{Classifying videos}
Videos are classified using the proposed method in Chapter \ref{chap:solution}, modified to predict only 2 classes: \textit{normal} and \textit{abnormal}. The classifier is trained 50 on real-life videos selected from the UCF-Crime\footnotetext{Source: \url{http://crcv.ucf.edu/cchen/}} dataset, as well as videos from the internet and achieved moderated result. As more and more video added for training, the classifier would expectantly classify video more accurately.
\begin{figure}[!htb]
	\minipage{0.5\textwidth}
	\includegraphics[width=\linewidth]{images/chap4/suspicious-2.png}
	\endminipage\hfill
	\minipage{0.05\textwidth}
	\endminipage\hfill
	\minipage{0.5\textwidth}
	\includegraphics[width=\linewidth]{images/chap4/suspicious-1.png}
	\endminipage
	\caption{Some abnormal videos for of training the classifier}
\end{figure}
\begin{figure}[!htb]
	\minipage{0.5\textwidth}
	\includegraphics[width=\linewidth]{images/chap4/normal-1.png}
	\endminipage\hfill
	\minipage{0.05\textwidth}
	\endminipage\hfill
	\minipage{0.5\textwidth}
	\includegraphics[width=\linewidth]{images/chap4/normal-2.jpg}
	\endminipage
	\caption{Some normal videos for training the classifier}
\end{figure}


\subsection{Environment}
The classifier system was deployed using Google Compute Service. This service allows users to host virtual computers with customizable hardware for many purposes. The deployed server runs on the Ubuntu operating system because of its high performance, stability and its compatibility with deep learning codes. The server has the following hardware specification:

\begin{itemize}
	\item \textbf{CPU}: Intel Xeon CPU @ 2.00GHz (4 cores) Skylake Architech
	\item \textbf{RAM}: 16GB ECC 
	\item \textbf{GPU}: NVIDIA K80 with 12GB VRAM
\end{itemize}
\textbf{NGINX} was used as the web server. It accepts any request from the default ports 80, 8080, then forward those request to a specified socket created by Gunicorn.\\
\textbf{Gunicorn} was used to serve the application. Gunicorn creates a socket so that web server could forward requests to, it then sends those request to the Python application. The Gunicorn process was set up as a system service so that it starts everytime the server boot up. Thec config file of this system service is in /etc/systemd/system/classifier-server.conf (Figure \ref{chap4:server_config}).
\begin{center}
    \begin{figure}[H]
    \centering
    \includegraphics[width=1\columnwidth]{images/chap4/server_config.png}
	\caption{The config of Gunicorn system service. The \textbf{ExecStart} variable defines command to start this process. It spawn a maximum of 2 workers to serve requests, and each worker executes 1 requests before resetting. A socket named classifier-server.sock  is created to listen incoming request to this service.}   
	\label{chap4:server_config}
    \end{figure}
\end{center}