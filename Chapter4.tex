\chapter{SYSTEM DESIGN AND IMPLEMENTATION}

This chapter describes the implementation of the system which consists of two subsystems: A social media website and an analysis server. In \textit{Requirement analysis and System overview} subsection, everything is briefly described as a whole. In three last sections, structure and details of two subsystems and the database are defined more comprehensively.

\section{Requirement specification}
A requirements analysis is essential to any system. In this section, this thesis defines functional as well as non-function requirements for the system. Use cases and a system overview is also outlined in the following subsections.
\subsection{Types of users}
As with any sophisticated system, the proposed system has many types of user. Those type of users are:

\begin{itemize}
	\item \textbf{Guest users}: Users that have not verify their account by authentication methods.
	\item \textbf{Verified users}: Users that are verified with the system by authentication methods.
	\item \textbf{Moderators}: Supervisors of the system. This title is given to people who are responsible for managing the system.
\end{itemize}
\subsection{Functional Requirements}
As stated in \ref{chap:solution}, this project would build a social media website for security. Such a website must have functions of a typical social media as well as additional functions for security. Those functions would include:
\begin{itemize}	
	\item Verified users and Guest users can view posts by other users. 
	\item Verified users could make posts with images or videos and includes location and time information in their posts. The website stores posts and image/videos in databases and analyze them through analyzing modules.
	\item Verified users could contribute to the system by labeling videos. Labels of videos are stored in the database and are used to train video classification modules.
	\item Verified users could subscribe to a location to get notifications about the suspicious videos from that location.
	\item When suspicious videos are detected, notifications are sent to Users subscribed to nearby locations.
	\item The system can recognize faces from images and logs their occurrences to the database.
\end{itemize}


\subsection{Non-functional Requirement}
In the scope of a thesis, the system has to acquire these requirements to perform processes that are stable and efficient for the user:
\begin{itemize}
	\item Reliability: The System is able to have 90\% uptime of Internet connection. It should have OAuth services via Google/Facebook(account kit) for authentication.
	\item Scalability: The website is able to handle many concurrent connections and manage a large amount of posts and store into NoSQL database .
	\item Usability: The website uses modern UI developed in HTML5. The user can use that UI easily to navigate between each page. It also has a fresh and well-designed interface for user to interact with
	      	
\end{itemize} 
\subsection{Use Cases}
\begin{center}
	\begin{figure}[H]
		\centering
		\includegraphics[width=0.75\columnwidth]{images/chap4/usecase.png}
		\caption{Use case diagram}
		\label{chap4:user_case_diagram}
	\end{figure}
\end{center}
\subsection{System overview}
Figure \ref{chap3:system_overview_basic} shows a concise overview of how the system operates. Users interact with the website through front-end. The input of users can come in the form of images, videos or label contribution. Back-end receives input and stores in the database. Concurrently, inputs are sent to analysis server . Outputs from analysis server are sent back to back-end and then stored in database. The back-end is also responsible for obtaining appropriate contents from the database to display to users through front-end.

\begin{center}
	\begin{figure}[H]
		\centering
		\includegraphics[width=1\columnwidth]{images/chap3/system_overview_basic.png}
		\caption{An overview of the system}
		\label{chap3:system_overview_basic}
	\end{figure}
\end{center}
\section{Database}
The project database divides into two different parts: \textbf{Cloud storage} for visual data such as image/video and a \textbf{NoSQL database} for other information. The primary target is to reduce website loading time. Let take an example, if files are stored directly on the back-end server. When many users request a file simultaneously, the server with limited bandwidth will cause delay. “53\% of mobile site visitors leave a page that takes longer than three seconds to load” – \href{https://think.storage.googleapis.com/docs/mobile-page-speed-new-industry-benchmarks.pdf}{Google}. If these files stored on cloud storage, the client will be served by the cloud storage provider, which has higher availability.
\subsection{Cloud storage}

\subsection{NoSQL database}
NoSQL databases were created in response to the limitations of traditional relational database technology. When compared against relational databases, NoSQL databases are more scalable and provide superior performance, and their data model addresses several shortcomings of the the relational model.

The advantages of NoSQL include being able to handle:

Large volumes of structured, semi-structured, and unstructured data
Agile sprints, quick iteration, and frequent code pushes
Object-oriented programming that is easy to use and flexible
Efficient, scale-out architecture instead of expensive, monolithic architecture
Hence, \textbf{MongoDB} is used to implement the database in this project
\subsubsection{Database overview}
Figure \ref{chap4:database_overview} displays the structure of the database which includes collections and relationships between them.
There are a total of 8 collections in the database: User, Visual data, Location, Notification, Person, Post, Record, Role. 
\begin{center}
	\begin{figure}[H]
		\centering
		\includegraphics[width=1\columnwidth]{images/chap4/Model.png}
		\caption{An overview of the database}
		\label{chap4:database_overview}
	\end{figure}
\end{center}
\cleardoublepage
\subsubsection{Collections}
\textbf{User}
\begin{center}
	\begin{figure}[H]
		\centering
		\includegraphics[width=0.5\columnwidth]{images/chap4/User.png}
		\caption{User collection}
	\end{figure}
\end{center}
\cleardoublepage
\textbf{Visual data}
\begin{center}
	\begin{figure}[H]
		\centering
		\includegraphics[width=1\columnwidth]{images/chap4/Visual.png}
		\caption{Visual Data collection}
	\end{figure}
\end{center}
\cleardoublepage
\textbf{Location}
\begin{center}
	\begin{figure}[H]
		\centering
		\includegraphics[width=1\columnwidth]{images/chap4/Location.png}
		\caption{Location collection}
	\end{figure}
\end{center}
\cleardoublepage
\textbf{Notification}
\begin{center}
	\begin{figure}[H]
		\centering
		\includegraphics[width=0.7\columnwidth]{images/chap4/Notification.png}
		\caption{Notification collection}
	\end{figure}
\end{center}
\cleardoublepage
\textbf{Person}
\begin{center}
	\begin{figure}[H]
		\centering
		\includegraphics[width=0.7\columnwidth]{images/chap4/Person.png}
		\caption{Person collection}
	\end{figure}
\end{center}
\cleardoublepage
\textbf{Post}
\begin{center}
	\begin{figure}[H]
		\centering
		\includegraphics[width=0.7\columnwidth]{images/chap4/Post.png}
		\caption{Post collection}
	\end{figure}
\end{center}
\cleardoublepage
\textbf{Record}
\begin{center}
	\begin{figure}[H]
		\centering
		\includegraphics[width=0.7\columnwidth]{images/chap4/Record.png}
		\caption{Record collection}
	\end{figure}
\end{center}
\cleardoublepage
\textbf{Role}
\begin{center}
	\begin{figure}[H]
		\centering
		\includegraphics[width=0.7\columnwidth]{images/chap4/Role.png}
		\caption{Role collection}
	\end{figure}
\end{center}
\cleardoublepage
\subsubsection{Relationships}
\begin{table}[H]
	\begin{tabular}{|l|l|}
		\hline
		\textbf{Children field}                             & \textbf{Parent field} \\ \hline
		Location.subscribers.{[}0{]}                        & User.\_id             \\ \hline
		Notification.to                                     & User.\_id             \\ \hline
		Notification.records.{[}0{]}                        & Record.\_id           \\ \hline
		Notification.data                                   & Visual Data.\_id      \\ \hline
		Notification.location                               & Location.\_id         \\ \hline
		Person.userCreated                                  & User.\_id             \\ \hline
		Person.datas.{[}0{]}                                & Visual Data.\_id      \\ \hline
		Person.location                                     & Location.\_id         \\ \hline
		Post.userCreated                                    & User.\_id             \\ \hline
		Post.datas.{[}0{]}                                  & Visual Data.\_id      \\ \hline
		Post.location                                       & Location.\_id         \\ \hline
		Post.reported.{[}0{]}                               & User.\_id             \\ \hline
		Record.data                                         & Visual data.\_id      \\ \hline
		Record.location                                     & Location.\_id         \\ \hline
		Record.personId                                     & Person.\_id           \\ \hline
		User.personId                                       & Person.\_id           \\ \hline
		User.address                                        & Location.\_id         \\ \hline
		User.subscribed.{[}0{]}                             & Location.\_id         \\ \hline
		Visual Data.labels.{[}0{]}.user                     & User.\_id             \\ \hline
		Visual Data.identifyResult.persons.{[}0{]}.personId & Person.\_id           \\ \hline
		Visual Data.location                                & Location.\_id         \\ \hline
		User.role                                           & Role.type             \\ \hline
	\end{tabular}
\end{table}
\cleardoublepage
Example:
\begin{table}[H]
	\begin{tabular}{|l|l|}
		\hline
		Children field               & Parent field \\ \hline
		Location.subscribers.{[}0{]} & User.\_id    \\ \hline
	\end{tabular}
\end{table}
The element of the field \textit{subscribers}(array) of collection \textbf{Location} references field \textit{\_id} of collection \textbf{User}.
\section{Social media website}
Model-View-Controller (MVC) pattern consists of three parts: Model, View, and Controller. This section will clarify each component.
\subsection{Front-end}
\subsection{Back-end}
In MVC, the Controller receives user input, manipulates Model, and then responds to the user (cite). Following is the list of implemented controllers in this thesis.
\begin{itemize}
	\item Authentication Controller: manages the flow of user sign in and sign up. A user can choose either traditional or social login. With the classical way, the user has to provide his/her email and password. If the user prefers social login, this website currently supports sign in via Facebook and Google+.
	\item Dataset Collector: downloads dataset (a collection of labeled videos) as the input for the training process.
	\item Email Controller: One way to interact with users is sending emails. This controller sends email using SendGrid, an email delivery service.
	\item Face Controller: The core of Face Recognition Module is Microsoft Azure Face API, a cognitive service that provides algorithms for recognizing human faces in images (cite). Face Controller contains a bunch of function to request Microsoft Azure Face API.
	\item Identify Controller
	\item Location Controller: 
	\item Notification Controller
	\item Person Controller
	\item Post Controller
	\item Record Controller
	\item Upload Controller
	\item User Controller
	\item Visual Data Controller
	      
\end{itemize}
\section{Analysis server}
The project requires two systems for analysis: A face recognition system and a video classifier system. The facial recognition system takes pictures of human faces as input and returns their identification. The video classifier system analyzes videos to find out actions in them. The remaining of this section describes in detail about the two analysis system.
\subsection{Face recognition module}
\subsection{Video classifier module}
\subsubsection{Deploying classifier system}
After training the model and saving the weights, to use the model for prediction, this thesis proposes building an HTTP server and provide an API for other components to send requests and receive the result. \\
Flask framework is used to implement this HTTP server because of its fast performance and is written in Python which helps communication with Keras APIs easier.
\begin{center}
    \begin{figure}[H]
    \centering
    \includegraphics[width=1\columnwidth]{images/chap4/server_sequence.png}
    \caption{Sequence diagram of the HTTP server}
    \end{figure}
\end{center}
\section{Technologies}
\subsection{Node.js and Node Package Manager (npm)}
Node.js (Node) is a JavaScript runtime built on Chrome's V8 JavaScript engine (cite). Node supports executing JavaScript on server-side. Ryan Dahl was created Node.js in 2009. Node.js Foundation is in charge of the development of Node.js. After nine years since its release, the latest LTS version of Node.js is 10.14.2 (includes npm 6.4.1). “Node.js operates on a single thread, using non-blocking I/O calls, allowing it to support tens of thousands (cite) of concurrent connections held in the event loop.” Although JavaScript is single-threaded, thanks to the Event loop, Node.js can implement asynchronous I/O operations. The Event loop will transfer operations to the system kernel. “Since most modern kernels are multi-threaded, they can handle multiple operations executing in the background. When one of these operations completes, the kernel tells Node.js so that the appropriate callback may be added to the poll queue to executed eventually”. By utilizes non-blocking I/O, Node.js skips the waiting time for I/O calls, which is much higher than processing time.
\subsection{Keras}
Keras is a high-level framework for building Deep learning model build on top of Tensorflow or Theano library.\\
Keras provides APIs that are more friendly for new learner compared to Tensorflow. Besides that, Keras also has useful functions for preprocessing data.
\subsection{Gunicorn}
Gunicorn is a Python Web Server Gateway Interface (WSGI) HTTP server. A WSGI recieve requests from web servers and foward them to Python applications.
Gunicorn helps separating the operation between web servers and applications, making applications more portable. Moreover, Gunicorn provides features for deployment such as maximum of requests for each worker, number of workers, making it useful for this thesis.
